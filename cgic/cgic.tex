%%%%%%%%%%%%%%%%%%%%%%%%%%%%%%%%%%%%%%%%%%%%%%%%%%%%%%%%%%%%%%%%%%%%%%%%%%%%%%
%
% Project main file
%
% Assumes that LaTeX2e macros and packages defined in cg_comp.sty are
%   available
%
%
%%%%%%%%%%%%%%%%%%%%%%%%%%%%%%%%%%%%%%%%%%%%%%%%%%%%%%%%%%%%%%%%%%%%%%%%%%%%%%
 \RequirePackage{pdf14}
 \pdfminorversion=6

\documentclass[12pt]{article}

\usepackage{cgic}

 \title{{\Huge\textbf{Classical Guitar Intonation and Compensation: The Well-Tempered Guitar}}}

 \author{ M.\ B.\ Anderson and R.\ G.\ Beausoleil \\
 \textit{Rosewood Guitar} \\
 \textit{8402 Greenwood Ave.\ N, Seattle, WA  98103}}

 \date{\today}

 \begin{document}

 \maketitle

 \begin{abstract}
Inspired by the pioneering work of luthier Greg Byers in 1996, we build an intuitive model of classical guitar intonation that includes the effects of the resonant length of the fretted string, linear mass density, tension, and bending stiffness. We begin by deriving an expression for the vibration frequencies of a stiff string using boundary conditions that are pinned at the saddle but clamped at the fret. Adopting logarithmic frequency differences based on ``cents'' that decouple these physical effects, we introduce Taylor series expansions that exhibit clearly the origins of frequency shifts of fretted notes from the corresponding Twelve-Tone Equal Temperament (12-TET) values. We demonstrate a simple \emph{in situ} technique for measurement of the changes in frequency of open strings arising from small adjustments in length, and we propose a simple procedure that any interested guitarist can use to estimate the corresponding shifts in frequency due to tension and bending stiffness for their own guitars and favorite string sets. Based on these results, we employ an RMS frequency error method to select values of saddle and nut setbacks that map fretted frequencies --- for a particular string set on a particular guitar --- almost perfectly onto their 12-TET targets. This exercise allow us to discuss a general approach to tempering an ``off-the-shelf'' guitar to further reduce the tonal errors inherent in any fretted musical instrument.
 \end{abstract}

 \tableofcontents

 %%%%%%%%%%%%%%%%%%%%%%%%%%%%%%%%%%%%%%%%%%%%%%%%%%%%%%%%%%%%%%%%%%%%%%%%%%%%%%
%
% Section file included in main project file using \input{}
%
% Assumes that LaTeX2e macros and packages defined in cg_comp.sty are
%   available
%
%%%%%%%%%%%%%%%%%%%%%%%%%%%%%%%%%%%%%%%%%%%%%%%%%%%%%%%%%%%%%%%%%%%%%%%%%%%%%%

 \section{Introduction and Background\label{sct:intro}}
Initial~\cite{ref:byersweb} and ongoing work by G.\ Byers\cite{ref:byers1996cgi}.

Recent studies of steel-string guitars~\cite{ref:varieschi2010icf}

 %%%%%%%%%%%%%%%%%%%%%%%%%%%%%%%%%%%%%%%%%%%%%%%%%%%%%%%%%%%%%%%%%%%%%%%%%%%%%%
%
% Section file included in main project file using \input{}
%
% Assumes that LaTeX2e macros and packages defined in cg_comp.sty are
%   available
%
%%%%%%%%%%%%%%%%%%%%%%%%%%%%%%%%%%%%%%%%%%%%%%%%%%%%%%%%%%%%%%%%%%%%%%%%%%%%%%

 \section{Simple Model of Guitar Intonation\label{sct:model}}

Fundamental frequency of a string~\cite{ref:morse1981vas,ref:morse1981vsa}:
 \begin{equation} \label{eqn:f_0_def}
f_0 = \frac{1}{2\, L_0}\, \sqrt{\frac{T_0}{\mu_0}}\, ,
 \end{equation}
where $L_0$ is the length of the free (unfretted) string from the saddle to the nut, $T_0$ is the tension in the free string, and $\mu_0 \equiv M / L_0$ is the linear mass density of a free string of mass $M$.

 \begin{equation} \label{eqn:f_0_stiff}
f_0 = \frac{1}{2\, L_0}\, \sqrt{\frac{T_0}{\mu_0}} \left[ 1 + B_0 + \left(1 + \frac{\pi^2}{8}\right) B_0^2 \right]\, ,
 \end{equation}
where $B_0$ is a ``string stiffness parameter.'' For a uniform string with a cylindrical cross section, $B_0$ given by~\cite{ref:morse1981vsb}
 \begin{equation} \label{eqn:b_def}
B_0 \equiv \sqrt{\frac{\pi\, R^4 E}{T_0\, L_0^2}}\, ,
 \end{equation}
where $R$ is the radius of the string and $E$ is Young's modulus (or the modulus of elasticity). For a typical nylon guitar string with $E \approx 2 - 4$~GPa, $T_0 \approx 50 - 70$~N, $R \approx 0.35 - 0.51$~mm, and $L_0 \approx 650$~mm, we have $B_0 \approx 0.007 - 0.026$, indicating that the corrections in \eqn{f_0_stiff} are not significant.

Throughout this work, we will use \emph{cents} to describe small differences in pitch~\cite{ref:durfee2015pms}. One cent is one one-hundredth of a 12-TET half step, so that there are 1200~cents per octave. The difference in pitch between frequencies $f_1$ and $f_2$ is therefore defined as
 \begin{equation} \label{eqn:cents_def}
\Delta \nu \equiv 1200\, \log_2\left(\frac{f_2}{f_1}\right)\, .
 \end{equation}
We define $f \equiv (f_1 + f_2) / 2$ and $\Delta f \equiv f_2 - f_1$. Then
 \begin{equation} \label{eqn:cents_approx}
\Delta \nu = 1200\, \log_2\left(\frac{f + \Delta f / 2}{f - \Delta f /2}\right) \approx \frac{1200}{\ln 2}\, \frac{\Delta f}{f}\, ,
 \end{equation}
where the last approximation applies when $\Delta f \ll f$. An experienced guitar player can distinguish beat notes with a difference frequency of $\Delta f \approx 1$~Hz, which corresponds to 8~cents at $A_2$ ($f = 220$~Hz) or 5~cents at $E_4$ ($f = 329.63$~Hz).

 \begin{figure}
  \centering
  \includegraphics[width=6.0in]{figures/guitar_schematic}
  \caption{\label{fig:guitar_schematic} A simple (side-view) schematic of the classical guitar used in this model. The scale length of the guitar is $X_0$, but we allow the edges of both the saddle and the nut to be set back an additional distance $\Delta S$ and $\Delta N$, respectively. The location on the $x$-axis of the center of the $n^\textrm{th}$ fret is $X_n$. In the $y$ direction, $y = 0$ is taken as the surface of the fingerboard; therefore the height of each fret above the fingerboard is $a$, the height of the nut is $a + b$, and the height of the saddle is $a + b + c$. $L_n$ is the \emph{resonant length} of the string from the saddle to the center of fret $n$, and $L^\prime_n$ is the length of the string from the fret to the nut.}
 \end{figure}

Our model begins with the schematic of the guitar shown in \fig{guitar_schematic}. The scale length of the guitar is $X_0$, but we allow the edges of both the saddle and the nut to be set back an additional distance $\Delta S$ and $\Delta N$, respectively. The location on the $x$-axis of the center of the $n^\textrm{th}$ fret is $X_n$. In the $y$ direction, $y = 0$ is taken as the surface of the fingerboard; the height of each fret is $a$, the height of the nut is $a + b$, and the height of the saddle is $a + b + c$. $L_n$ is the \emph{resonant length} of the string from the saddle to the center of fret $n$, and $L^\prime_n$ is the length of the string from the fret to the nut. The total length of the string is defined as $\mathcal{L}_n \equiv L_n + L^\prime_n$. For reasons discussed below, we have not adopted a more complicated fretting model~\cite{ref:byers1996cgi,ref:varieschi2010icf}. We start with the simple form of the fundamental frequency of a string given by \eqn{f_0_def}, and apply it to the frequency of a string pressed just behind the $n^\mathrm{th}$ fret:
 \begin{equation} \label{eqn:f_n_def}
f_n = \frac{1}{2\, L_n}\, \sqrt{\frac{T_n}{\mu_n}}\, ,
 \end{equation}
where $T_n$ and $\mu_n$ are respectively the tension and linear mass density of the fretted string. We note that $T_n$ and $\mu_n$ depend on $\mathcal{L}_n$, the \emph{total} length of the fretted string from the saddle to the nut. Ideally, in the 12-TET equal-temperament system~\cite{ref:durfee2015pms},
 \begin{equation} \label{eqn:f_n_tet}
f_n = \gamma_n\, f_0\, , \qquad \textrm{(12-TET~ideal)}
 \end{equation}
where
 \begin{equation} \label{eqn:gamme_n_def}
\gamma_n \equiv 2^{n / 12}\, .
 \end{equation}
Therefore, the error interval expressed in cents is given by
 \begin{equation}\label{eqn:error_def}
 \begin{split}
\Delta \nu_n &= 1200\, \log_2\left( \frac{f_n}{\gamma_n\, f_0} \right) \\
&= 1200\, \log_2 \left( \frac{L_0}{\gamma_n\, L_n}\, \sqrt{\frac{\mu_0}{\mu_n}\, \frac{T_n}{T_0}}\, \right) \\
&= 1200\, \log_2 \left( \frac{L_0}{\gamma_n\, L_n} \right) + 600\, \log_2 \left(  \frac{\mu_0}{\mu_n} \right) + 600\, \log_2 \left( \frac{T_n}{T_0} \right)\, .
 \end{split}
 \end{equation}

The final form of \eqn{error_def} makes it clear that --- for nylon guitar strings --- there are three contributions to intonation:
 \begin{enumerate}
  \item
   \emph{Resonant Length}: The first term represents the error caused by the increase in the length of the fretted string $L_n$ compared to the ideal length $X_n$, which would be obtained if $b = c = 0$ and $\Delta S = \Delta N = 0$.
  \item
   \emph{Linear Mass Density}: The second term is the error caused by the reduction of the linear mass density of the fretted string. This effect will depend on the \emph{total} length of the string, given by $\mathcal{L}_n = L_n + L^\prime_n$.
  \item
   \emph{Tension}: The third (and most complex) term is the error caused by the \emph{increase} of the tension in the string caused by the stress and strain applied to the string by fretting. This effect will also depend on the total length of the string $\mathcal{L}_n$.
 \end{enumerate}
We will discuss each of these three sources of error in turn below.

 \subsection{Resonant Length Error}
We can estimate the first term in the last line of \eqn{error_def} by referring to \fig{guitar_schematic} and computing the resonant length $L_n$. We find:
     \begin{equation}
L_n = \begin{cases}
\sqrt{\left(X_0 + \Delta S + \Delta N\right)^2 + c^2}\, , & n =  0 \\
\sqrt{\left(X_n + \Delta S\right)^2 + (b + c)^2}\, . & n \ge 1
      \end{cases}
     \end{equation}
When $b + c \ll X_0$ and the guitar has been manufactured such that $X_n = X_0 / \gamma_n$, we can approximate $L_n$ by
    \begin{equation}
L_n \approx \begin{cases}
X_0 + \Delta S + \Delta N + c^2/2\, X_0\, , & n =  0 \\
\gamma_n^{-1} \left\{X_0 + \gamma_n \Delta S + \left[\gamma_n (b + c)\right]^2/2\, X_0\right\}\, . & n \ge 1
      \end{cases}
     \end{equation}
 Suppose that $X_0 = 650$~mm, $b + c = 5$~mm, and $n = 12$ (i.e., $\gamma_{12} = 2$). Then $\left[\gamma_n (b + c)\right]^2/2\, X_0 < 0.1$~mm, and it's clear that we can neglect the terms here that depend on $b$ and/or $c$. Then the resonant length error is approximately
  \begin{equation}
  1200\, \log_2 \left( \frac{L_0}{\gamma_n\, L_n} \right) \approx -\frac{1200}{\ln(2)\, X_0} \left[\left(\gamma_n - 1\right) \Delta S - \Delta N\right]
  \end{equation}
If the guitar is uncompensated, so that $\Delta S = \Delta N = 0$, this error is negligible. In fact, for most classical guitars and strings, it is much less than 1~cent because the impact of $b$ and $c$ on $L_n$ is quadratic and therefore very small. But, with $\Delta S > 0$ and $\Delta N < 0$, we can \emph{increase} the magnitude of this ``error'' and cause the frequency to shift lower. We'll see that this is our primary method of compensation.

Previous studies of guitar intonation and compensation have chosen to include the apparent increase in length of the string caused by both the depth of the fretting action \red{(Matt: Is this the right term?)} and the shape of the fretted string under the finger~\cite{ref:byers1996cgi,ref:varieschi2010icf}. As the string is initially pressed to the fret, the total length $\mathcal{L}_n$ increases and causes the tension in the string --- which is clamped at the saddle and the nut --- to increase. As the string is pressed further, does the additional deformation of the string increase its tension (throughout the resonant length $L_n$)? There are at least two purely empirical reasons to doubt this hypothesis. First, we can mark a string (with a fine-point felt pen) above a particular fret and then observe the mark with a magnifying glass. As the string is pressed all the way to the finger board, the mark does not move perceptibly --- it has become effectively \emph{clamped} on the fret. Second, we can use either our ears or a simple tool to measure frequencies~\cite{ref:pgtweb} to listen for a shift as we use different fingers and vary the fretted depth of a string. The apparent modulation is far less than would be obtained by classical vibrato, so we assume that once the string is minimally fretted the length(s) can be regarded as fixed. (If this were not the case, then fretting by different people or with different fingers, at a single fret or using a barre, would cause additional, varying frequency shifts that would be audible and difficult to compensate.)

 %%%%%%%%%%%%%%%%%%%%%%%%%%%%%%%%%%%%%%%%%%%%%%%%%%%%%%%%%%%%%%%%%%%%%%%%%%%%%%
%
% Section file included in main project file using \input{}
%
% Assumes that LaTeX2e macros and packages defined in cg_comp.sty are
%   available
%
%%%%%%%%%%%%%%%%%%%%%%%%%%%%%%%%%%%%%%%%%%%%%%%%%%%%%%%%%%%%%%%%%%%%%%%%%%%%%%

 \section{Experimental Estimate of the String Constant\label{sct:exp}}

 \begin{equation}
 \begin{split}
L_{n \ge 1}(y) &= \sqrt{\left(X_n + \Delta S\right)^2 + (b + c)^2 + y^2} \\
&\approx X_n + \Delta S + \frac{(b + c)^2 + y^2}{2\, X_n}\, .
 \end{split}
 \end{equation}

 \begin{equation}
 \begin{split}
L^\prime_{n \ge 1}(y) &= \sqrt{\left(X_0 - X_n + \Delta N\right)^2 + b^2 + y^2} \\
&\approx X_0 - X_n + \Delta N + \frac{b^2 + y^2}{2 \left(X_0 - X_n\right)}\, .
 \end{split}
 \end{equation}

 \begin{equation}
 \begin{split}
\lambda_n(y) &\approx \frac{1}{2\, X_0} \left[ \frac{(b + c)^2 + y^2}{X_n} + \frac{b^2 + y^2}{X_0 - X_n} - \frac{c^2}{X_0} \right] \\
&= \lambda_n(0) + \Delta \lambda_n(y) \, ,
 \end{split}
 \end{equation}
where $\lambda_n(0)$ is given by \eqn{lambda_n_approx}, and
 \begin{equation}
\Delta \lambda_n(y) \equiv \frac{1}{2 \left(\gamma_n - 1\right)}\, \left(\frac{\gamma_n\, y}{X_0}\right)^2\, .
 \end{equation}

Following the same approach we used to derive \eqn{quad_shift}, we can derive the change in the total shift due to both resonant length and linear mass density for a transverse displacement $y$. To second order in $y$, we find that
 \begin{equation}
\Delta \nu_n(y) \approx \frac{600}{\ln(2)}\, \frac{3 - 2 \gamma_n}{2 \left(\gamma_n - 1\right)}\, \left(\frac{\gamma_n\, y}{X_0}\right)^2\, .
 \end{equation}
We have plotted this expression for the first 12 frets and $y = 5$~mm in \fig{quad_shift_factory}. This shift is quite small compared to the experimental errors we'll obtain in the shifts due to tension, and we ignore it in what follows.

 \begin{figure}
  \centering
  \includegraphics[width=5.0in]{figures/quad_shift_factory}
  \caption{\label{fig:quad_shift_factory} Total frequency shift (in cents) due to resonant length and linear mass density for a transverse displacement of $y = 5$~mm. This shift is identical for each string, and should be smaller than the experimental errors we'll accumulate using our transverse displacement approach.}
 \end{figure}


% \begin{figure}
%  \centering
%  \begin{subfigure}[b]{0.8\textwidth}
%   \centering
%   \includegraphics[width=5.0in]{figures/norm_error_uncompensated}
%   \caption{Frequency shift for an uncompensated guitar}
%   \label{fig:norm_error_uncompensated}
%  \end{subfigure}
%  \par\vspace{0.25in}
%  \begin{subfigure}[b]{0.8\textwidth}
%   \centering
%   \includegraphics[width=5.0in]{figures/norm_error_factory}
%   \caption{Frequency shift for a factory guitar}
%   \label{fig:norm_error_factory}
%  \end{subfigure}
%  \caption{\label{fig:norm_error} Frequency shift (in cents) due to the fretted length $L_n$ for an uncompensated (a) and factory (b) Alhambra 8P guitar, for both zero and nonzero lateral displacement $y$.}
% \end{figure}
%
 \begin{figure}
  \centering
  \begin{subfigure}[b]{0.8\textwidth}
   \centering
   \includegraphics[width=5.0in]{figures/shift_data}
   \caption{Experimental data}
   \label{fig:shift_data}
  \end{subfigure}
  \par\vspace{0.25in}
  \begin{subfigure}[b]{0.8\textwidth}
   \centering
   \includegraphics[width=5.0in]{figures/delta_lambda}
   \caption{Calculated change in total string length $\mathcal{L}$}
   \label{fig:delta_l}
  \end{subfigure}
  \caption{\label{fig:exp_data} Frequency shift (in cents) (a) and change in total string length $\mathcal{L}$ (b) due to lateral displacement $y$.}
 \end{figure}

 %%%%%%%%%%%%%%%%%%%%%%%%%%%%%%%%%%%%%%%%%%%%%%%%%%%%%%%%%%%%%%%%%%%%%%%%%%%%%%
%
% Section file included in main project file using \input{}
%
% Assumes that LaTeX2e macros and packages defined in cg_comp.sty are
%   available
%
%%%%%%%%%%%%%%%%%%%%%%%%%%%%%%%%%%%%%%%%%%%%%%%%%%%%%%%%%%%%%%%%%%%%%%%%%%%%%%

 \section{Classical Guitar Compensation\label{sct:comp}}

 \begin{table}[htbp]
  \centering
  \caption{\label{tbl:ej45_setbacks} Predicted setbacks for the D'Addario Pro-Arte Nylon Classical Guitar Strings -- Normal Tension (EJ45) on the Alhambra 8P classical guitar.}
    \begin{tabular}{lcccc}
    \hline \hline
    String  & $\Delta S$~(mm) & $\Delta N$~(mm) \\
    \hline
    J4301 & 2.15 & -0.44 \\
    J4302 & 1.90 & -0.32 \\
    J4303 & 4.30 & -0.86 \\
    J4304 & 1.29 & -0.19 \\
    J4305 & 1.93 & -0.30 \\
    J4306 & 2.59 & -0.37 \\
    \hline \hline
    Mean & 2.36 & -0.41 \\
    \hline
    \end{tabular}%
 \end{table}%

 \begin{figure}
  \centering
  \begin{subfigure}[b]{0.45\textwidth}
   \centering
   \includegraphics[width=3.25in]{figures/shift_alhambra8p_ej45_null}
   \caption{Uncompensated}
   \label{fig:shift_alhambra8p_ej45_null}
  \end{subfigure}
  \hspace{0.25in}
  \begin{subfigure}[b]{0.45\textwidth}
   \centering
   \includegraphics[width=3.25in]{figures/shift_alhambra8p_ej45_factory}
   \caption{Factory guitar}
   \label{fig:shift_alhambra8p_ej45_factory}
  \end{subfigure}
  \par\vspace{0.25in}
  \begin{subfigure}[b]{0.45\textwidth}
   \centering
   \includegraphics[width=3.25in]{figures/shift_alhambra8p_ej45_full}
   \caption{Full compensation}
   \label{fig:shift_alhambra8p_ej45_full}
  \end{subfigure}
  \hspace{0.25in}
  \begin{subfigure}[b]{0.45\textwidth}
   \centering
   \includegraphics[width=3.25in]{figures/shift_alhambra8p_ej45_mean}
   \caption{Mean compensation}
   \label{fig:shift_alhambra8p_ej45_mean}
  \end{subfigure}
  \caption{\label{fig:compensation_alhambra8p_ej45} Frequency shift (in cents) for an Alhambra 8P guitar with D'Addario Pro-Arte Nylon Classical Guitar Strings -- Normal Tension (EJ45). Four different strategies of saddle and nut compensation are illustrated.}
 \end{figure}


 %%%%%%%%%%%%%%%%%%%%%%%%%%%%%%%%%%%%%%%%%%%%%%%%%%%%%%%%%%%%%%%%%%%%%%%%%%%%%%
%
% Section file included in main project file using \input{}
%
% Assumes that LaTeX2e macros and packages defined in cg_comp.sty are
%   available
%
%%%%%%%%%%%%%%%%%%%%%%%%%%%%%%%%%%%%%%%%%%%%%%%%%%%%%%%%%%%%%%%%%%%%%%%%%%%%%%

 \section{Tempering the Classical Guitar\label{app:temp}}

 \begin{figure}
  \centering
  \begin{subfigure}[b]{0.45\textwidth}
   \centering
   \includegraphics[width=3.25in]{figures/shift_alhambra8p_ej45_factory}
   \caption{Factory guitar --- 12-TET tuned}
   \label{fig:shift_alhambra8p_ej45_fact_temp}
  \end{subfigure}
  \hspace{0.25in}
  \begin{subfigure}[b]{0.45\textwidth}
   \centering
   \includegraphics[width=3.25in]{figures/shift_alhambra8p_ej45_harmonic}
   \caption{Factory guitar --- harmonically tuned}
   \label{fig:shift_alhambra8p_ej45_harmonic}
  \end{subfigure}
  \caption{\label{fig:compensation_alhambra8p_ej45_temp} Frequency shift (in cents) for an Alhambra 8P guitar with D'Addario Pro-Arte Nylon Classical Guitar Strings -- Normal Tension (EJ45). Here we compare the factory guitar tuned to 12-TET with the same guitar harmonically tuned.}
 \end{figure}

 %%%%%%%%%%%%%%%%%%%%%%%%%%%%%%%%%%%%%%%%%%%%%%%%%%%%%%%%%%%%%%%%%%%%%%%%%%%%%%
%
% Section file included in main project file using \input{}
%
% Assumes that LaTeX2e macros and packages defined in cg_comp.sty are
%   available
%
%%%%%%%%%%%%%%%%%%%%%%%%%%%%%%%%%%%%%%%%%%%%%%%%%%%%%%%%%%%%%%%%%%%%%%%%%%%%%%

 \section{Conclusion: The Recipe\label{sct:conc}}

 In this work, we have constructed a model of classical guitar intonation that includes the effects of the resonant length of the fretted string, linear mass density, tension, and bending stiffness. We have described a simple experimental approach to estimating the increase in string tension arising from an increase in its length, and then the corresponding mechanical stiffness. This allows us to determine the saddle and nut positions needed to compensate the guitar for a particular string, and we propose a simple approach to find averages of these positions to accommodate a variety of strings. This ``mean'' method benefits further from temperament techniques --- such as harmonic tuning --- that can enhance the intonation of the classical guitar for particular musical pieces.

Our calculations have relied on \eqn{f_m_stiff}, which was derived by noting that the string was pinned to the saddle rather than clamped. We then separated the contributions to the frequency deviations from ideal values caused by fretting by measuring these differences using the logarithmic ``cents'' approach defined by \eqn{cents_def}. This approach has allowed us to determine Taylor series approximations to each of these contributions that are valid when the height of a string above each fret is small compared to the scale length. From these approximations, we have been able to create a simple arithmetic ``recipe'' that predicts saddle and nut setbacks that enable the guitar to compensate for the frequency effects  of fretting:
\begin{enumerate}
    \item Determine the scale length of the guitar by doubling the distance between the inside of the nut and the center of the $12^\textrm{th}$ fret. 
    \item Carefully measure the values of $b$ and $c$. It is possible that the luthier has selected a saddle with vertical curvature, resulting in different values of $c$ for each string.
    \item Estimate the relief $\Delta y_{12}$ at the $12^\textrm{th}$ fret. Measure the action (height) $y_{12}$ of the string above fret 12; then $\Delta y_{12} = y_{12} - b - c/2$.
    \item Select a string set with values of $\kappa$ and $B_0$ listed in one of the derived physical properties tables in this paper, or follow the procedure developed in \sct{exp} to determine these quantities for a different string set.
    \item Compute the mean value of $Q$ using \eqn{qbar_approx}:
    \begin{equation*}
        \overline{Q} \approx \frac{77.9\, b^2 + 35.6\, b\, \left(c + 2\, \Delta y_{12}\right) + 5.82\, \left(c + 2\, \Delta y_{12}\right)^2}{24\, X_0^2}\, .
    \end{equation*}
    \item Compute the setbacks using \eqn{comp_approx}:
    \begin{align*}
        \Delta S &= B_0\, X_0\, , \nd \\
        \Delta N &= - \kappa\, X_0\, \overline{Q} / 2\, ,
    \end{align*}
\end{enumerate}
These setback estimates can be averaged across the string set to design compensated nuts and saddles that should be relatively easy to fabricate. Note that the resulting mean values are reasonably close to the results we'd obtain using the more exact approach presented in \sct{comp} and \app{rms} for a fretting distance $d = 10$~mm. Nevertheless, we understand that high-end (concert) guitars that are likely to rely on one or two string sets (and the appropriate value of $d$ for one guitar player) will benefit from the full, more accurate treatment of individual string setbacks.

We have placed the text of this manuscript (as well as the computational tools needed to reproduce our numerical results and all of the graphs presented here) online~\cite{ref:github2021rgb} so as to invite comment and contributions from and collaboration with interested luthiers and musicians.

 \appendix
 %%%%%%%%%%%%%%%%%%%%%%%%%%%%%%%%%%%%%%%%%%%%%%%%%%%%%%%%%%%%%%%%%%%%%%%%%%%%%%
%
% Appendix file included in main project file using \input{}
%
% Assumes that LaTeX2e macros and packages defined in cg_comp.sty are
%   available
%
%%%%%%%%%%%%%%%%%%%%%%%%%%%%%%%%%%%%%%%%%%%%%%%%%%%%%%%%%%%%%%%%%%%%%%%%%%%%%%

 \section{Fretting Model\label{app:fret}}

 \begin{equation}
L^\prime_n = d + \sqrt{\left(X_0 + \Delta N - X_n - d\right)^2 + b^2} \approx X_0 - X_n + \Delta N + \frac{b^2}{2 \left(X_0 + \Delta N - X_n - d\right)}\, ,
 \end{equation}


 \begin{figure}
  \centering
  \includegraphics[width=6.0in]{figures/fret_model}
  \caption{\label{fig:fret_model} Plot of the normalized displacement $Q_n$ as a function of the fret number for three different values of the parameter $d$. Here the guitar has $b = 1.0$~mm, $c = 3.5$~mm, no setbacks, and a scale length of 650~mm.}
 \end{figure}




 %%%%%%%%%%%%%%%%%%%%%%%%%%%%%%%%%%%%%%%%%%%%%%%%%%%%%%%%%%%%%%%%%%%%%%%%%%%%%%
%
% Section file included in main project file using \input{}
%
% Assumes that LaTeX2e macros and packages defined in cg_comp.sty are
%   available
%
%%%%%%%%%%%%%%%%%%%%%%%%%%%%%%%%%%%%%%%%%%%%%%%%%%%%%%%%%%%%%%%%%%%%%%%%%%%%%%

 \section{Vibration Frequencies of a Stiff String\label{app:freq}}

Here we outline the calculation of the normal mode frequencies of a vibrating stiff string with non-symmetric boundary conditions. We begin with the wave equation~\cite{ref:fletcher1964nvf}
 \begin{equation}
\mu\, \frac{\partial^2}{\partial t^2}\, y(x) = T\, \frac{\partial^2}{\partial x^2}\, y(x) - E\, S\, \mathcal{K}^2\, \frac{\partial^4}{\partial x^2}\, y (x)\, ,
 \end{equation}
where $\mu$ and $T$ are respectively the linear mass density and the tension of the string, $E$ is its Young's modulus (or the modulus of elasticity), $S$ is the cross-sectional area, and $\mathcal{K}$ is the radius of gyration of the string. If we simplify this equation by scaling $x$ by the length $L$ of the string, and $t$ by $1/\omega_0 \equiv  (L/\pi) \sqrt{\mu/T}$, then we obtain the dimensionless wave equation
 \begin{equation} \label{eqn:wave_eqn_dim}
\pi^2\, \frac{\partial^2}{\partial t^2}\, y(x) = \frac{\partial^2}{\partial x^2}\, y(x) - B^2\, \frac{\partial^4}{\partial x^2}\, y (x)\, ,
 \end{equation}
where $B$ is the ``bending stiffness parameter'' given by
 \begin{equation} 
B \equiv \sqrt{\frac{E\, S\, \mathcal{K}^2}{L^2 T}}\, .
 \end{equation}
If we assume that $y(x)$ is a sum of terms of the form
 \begin{equation}
y(x) = \mathcal{C}\, e^{k\, x - i\, \omega\, t}\, ,
 \end{equation}
then $k$ and $\omega$ must satisfy the expression
 \begin{equation}
B^2 k^4 - k^2 - (\pi\, \omega)^2 = 0\, ,
 \end{equation}
or
 \begin{equation}
k^2 = \frac{1 \pm \sqrt{1 + (2\, \pi\, B\, \omega)^2}}{2\, B^2}\, .
 \end{equation}
Therefore, given $\omega$, we have four possible choices for $k$: $\pm k_1$, or $\pm i k_2$, where
 \begin{subequations} \label{eqn:disp_eqns}
 \begin{align}
\label{eqn:disp_eqn_1} k_1^2 &= \frac{\sqrt{1 + (2\, \pi\, B\, \omega)^2} + 1}{2\, B^2}\, , \nd \\
\label{eqn:disp_eqn_2} k_2^2 &= \frac{\sqrt{1 + (2\, \pi\, B\, \omega)^2} - 1}{2\, B^2}\, .
 \end{align}
 \end{subequations}
Therefore, the general solution to \eqn{wave_eqn_dim} has the form
 \begin{equation} \label{eqn:soln_dim}
y(x) = e^{-i \omega t} \left( C_1^+ e^{k_1 x} + C_1^- e^{-k_1 x} + C_2^+ e^{i k_2 x} + C_2^- e^{-i k_2 x} \right)\, .
 \end{equation}

As discussed in \sct{model}, the boundary conditions for the case of a classical guitar string are not symmetric. At $x = 0$ (the saddle), the string is pinned (but not clamped), so that $y = 0$ and $\partial^2 y/\partial x^2 = 0$. However, at $x = 1$ (the fret) the string is clamped, so that $y = 0$ and $\partial y/\partial x = 0$. Applying these boundary conditions to \eqn{soln_dim}, we obtain
 \begin{subequations}
 \begin{align}
  0 &= C_1^+ + C_1^- + C_2^+ + C_2^-\, , \\
  0 &= k_1^2 \left(C_1^+ + C_1^-\right) - k_2^2 \left(C_2^+ + C_2^-\right)\, , \\
  0 &= C_1^+ e^{k_1} + C_1^- e^{-k_1} + C_2^+ e^{i k_2} + C_2^- e^{-i k_2}\, , \nd \\
  0 &= k_1 \left(C_1^+ e^{k_1} - C_1^- e^{-k_1}\right) + i k_2 \left(C_2^+ e^{i k_2} - C_2^- e^{-i k_2}\right)\, .
 \end{align}
 \end{subequations}
Since $k_1^2 + k_2^2 \ne 0$, the first two of these equations tell us that $C_1^- = -C_1^+ \equiv -C_1$, and $C_2^- = -C_2^+ \equiv -C_2$. Therefore, the second two equations become
 \begin{subequations}
 \begin{align}
C_1\, \sinh(k_1) &= - i\, C_2\, \sin(k_2)\, , \nd \\
k_1\, C_1\, \cosh(k_1) &= -i\, k_2\, C_2\, \cos(k_2)\, .
 \end{align}
 \end{subequations}
Dividing the first of these equations by the second, we find
 \begin{equation}
\tan(k_2) = \frac{k_2}{k_1}\, \tanh{k_1}\, .
 \end{equation}

In the case of a classical guitar, we expect that $B \ll 1$, so from \eqn{disp_eqn_1} we see that $k_1 \approx 1/B \gg 1$, and therefore $\tanh{k_1} \longrightarrow 1$. Writing $k_2 = m \pi (1 + \epsilon)$, where $m$ is an integer greater than or equal to 1, and $\epsilon \ll 1$, we have $\tan(k_2) \approx m \pi \epsilon$, and
 \begin{equation}
\frac{k_2}{k_1} \approx m \pi B (1 + \epsilon) = m \pi \epsilon\, ,
 \end{equation}
indicating that $\epsilon = B$ to lowest order in $B$. To lowest (zero) order in $B$, we see from \eqn{disp_eqn_2} that $k_2 = \pi \omega = m \pi (1 + B)$, or $\omega = m (1 + B)$. Restoring the time scaling by $1/\omega_0$, and defining the frequency $f = \omega/2 \pi$, we finally have
 \begin{equation} \label{eqn:f_m_stiff}
f_m = \frac{m}{2\, L}\, \sqrt{\frac{T}{\mu}} (1 + B)\, .
 \end{equation}
We use this result to build our model in \sct{model}.
 %%%%%%%%%%%%%%%%%%%%%%%%%%%%%%%%%%%%%%%%%%%%%%%%%%%%%%%%%%%%%%%%%%%%%%%%%%%%%%
%
% Appendix file included in main project file using \input{}
%
% Assumes that LaTeX2e macros and packages defined in cg_comp.sty are
%   available
%
%%%%%%%%%%%%%%%%%%%%%%%%%%%%%%%%%%%%%%%%%%%%%%%%%%%%%%%%%%%%%%%%%%%%%%%%%%%%%%

 \section{Compensation by Minimizing RMS Error\label{app:rms}}

The root-mean-square (RMS) frequency error (in cents) averaged over the frets $n \in \{1, n_\text{max}\}$ \emph{of a particular string} is given by
 \begin{equation}\label{eqn:rms_def}
\overline{\Delta \nu}_\text{rms} \equiv \sqrt{\frac{\sum_{n = 1}^{n_\text{max}} \Delta \nu_n^2}{n_\text{max}}}\, ,
 \end{equation}
where $\Delta \nu_n$ is given by \eqn{error_tot}. Here we will vary both $\Delta S$ and $\Delta N$ to minimize $\overline{\Delta \nu}_\text{rms}$. In this case, it is sufficient to minimize the quantity
 \begin{equation}\label{eqn:chi2_def}
\chi^2 = \sum_{n = 1}^{n_\text{max}} \left[\frac{\ln(2)}{1200}\, \Delta \nu_n\right]^2
 \end{equation}
such that the gradient of $\chi^2$ with respect to $\Delta S$ and $\Delta N$ vanishes. The components of this gradient are
 \begin{subequations}\label{chi2_grad}
 \begin{align}
\frac{\partial}{\partial \Delta S}\, \chi^2 &= -\frac{2}{X_0}\, \sum_n (\gamma_n - 1)\left[ (\gamma_n - 1) \left(B_0 - \frac{\Delta S}{X_0}\right) + \frac{\Delta N}{X_0} + \frac{\kappa}{2}\, Q_n \right]\, , \nd \\
\frac{\partial}{\partial \Delta N}\, \chi^2 &= \frac{2}{X_0}\, \sum_n \left[ (\gamma_n - 1) \left(B_0 - \frac{\Delta S}{X_0}\right) + \frac{\Delta N}{X_0} + \frac{\kappa}{2}\, Q_n \right]\, .
 \end{align}
 \end{subequations}
Setting both of these expressions to zero, we can rewrite them as the matrix equation
 \begin{equation} \label{eqn:rms_mat}
\begin{bmatrix}
  \sigma_2 & -\sigma_1 \\
  \sigma_1 & -\sigma_0
\end{bmatrix}\,
\begin{bmatrix}
  \Delta S \\
  \Delta N
\end{bmatrix} = X_0
\begin{bmatrix}
  \sigma_2\, B_0 +  \half\, \kappa\, \overline{Q}_1 \\
  \sigma_1\, B_0 +  \half\, \kappa\, \overline{Q}_0
\end{bmatrix}\, ,
 \end{equation}
where
 \begin{align}
\sigma_k &\equiv \sum_{n = 1}^{n_\text{max}} \left(\gamma_n - 1\right)^k\, , \nd \\
\overline{Q}_k &\equiv \sum_{n = 1}^{n_\text{max}} \left(\gamma_n - 1\right)^k\, Q_n\, .
 \end{align}
We note that
\begin{subequations}
  \begin{align}
    \sigma_0 &= n_\text{max}\, ,\\
    \sigma_1 &= \frac{\gamma_1 \left(\gamma_{n_\text{max}} - 1\right)}{\gamma_1 - 1} - \sigma_0
  \end{align}
\end{subequations}

 \Eqn{rms_mat} has the straightforward analytic solution
\begin{equation}
  \begin{bmatrix}
    \Delta S \\
    \Delta N
  \end{bmatrix} = \frac{X_0}{\sigma_1^2 - \sigma_0\, \sigma_2}\,
  \begin{bmatrix}
    -\sigma_0 & \sigma_1 \\
    -\sigma_1 & \sigma_2
  \end{bmatrix}\,
  \begin{bmatrix}
    \sigma_2\, B_0 + \half\, \kappa\, \overline{Q}_1 \\
    \sigma_1\, B_0 + \half\, \kappa\, \overline{Q}_0
  \end{bmatrix}\, ,
\end{equation}
or
\begin{subequations} \label{eqn:rms_sol}
  \begin{align}
    \label{eqn:rms_sol_ds} \Delta S &= B_0\, X_0 + \frac{\kappa\, X_0}{2 \left(\sigma_1^2 - \sigma_0\, \sigma_2\right)}\, \left( \sigma_1\, \overline{Q}_0 - \sigma_0\, \overline{Q}_1 \right)\, , \nd \\
    \label{eqn:rms_sol_dn} \Delta N &= \frac{\kappa\, X_0}{2 \left(\sigma_1^2 - \sigma_0\, \sigma_2\right)}\, \left( \sigma_2\, \overline{Q}_0 - \sigma_1\, \overline{Q}_1 \right)\, .
  \end{align}
\end{subequations}
For completeness, the corresponding solution when the quadratic stiffness term is included is given by
 \begin{equation}\label{eqn:rms_sol_quad}
\begin{bmatrix}
  \Delta S \\
  \Delta N
\end{bmatrix} = \frac{X_0}{\sigma_1^2 - \sigma_0\, \sigma_2}\,
\begin{bmatrix}
  -\sigma_0 & \sigma_1 \\
  -\sigma_1 & \sigma_2
\end{bmatrix}\,
\begin{bmatrix}
  \sigma_2\, B_0 + \half \left(1 + \pi^2\right) \left(2 \sigma_2 + \sigma_3\right) B_0^2 + \half\, \kappa\, \overline{Q}_1 \\
  \sigma_1\, B_0 + \half \left(1 + \pi^2\right) \left(2 \sigma_1 + \sigma_2\right) B_0^2 + \half\, \kappa\, \overline{Q}_0
\end{bmatrix}\, .
 \end{equation}

The corresponding Hessian matrix for this problem is
 \begin{equation}
H = \begin{bmatrix}
      \frac{\partial^2 \chi^2}{\partial \Delta S^2} & \frac{\partial^2 \chi^2}{\partial \Delta N\, \partial \Delta S} \\
      \frac{\partial^2 \chi^2}{\partial \Delta S\, \partial \Delta N} & \frac{\partial^2 \chi^2}{\partial \Delta N^2}
    \end{bmatrix}
  = \frac{2}{X_0^2} \begin{bmatrix}
      \sigma_2 & -\sigma_1 \\
      -\sigma_1 & \sigma_0
    \end{bmatrix}\, .
 \end{equation}
The Hessian is positive definite if and only if all of its eigenvalues are positive, and in the case of a $2 \times 2$ real matrix, this holds when the determinant is greater than zero. It is easy to verify numerically (and with some effort algebraically) that $\Det(H) > 0$ for $n_\text{max} > 1$. Therefore, the solution for $\Delta S$ and $\Delta N$ given by \eqn{rms_sol} minimizes the RMS frequency error.

The setback solution given by \eqn{rms_sol} is valid for a single string, and results like those shown in \tbl{ej45_setbacks} and \fig{shift_alhambra8p_ej45_full} assume that the guitar is built such that each string --- from a particular set of strings --- has a unique saddle and nut setback. Suppose that we'd prefer to engineer a guitar with single, uniform values of both $\Delta S$ and $\Delta N$ that provide reasonable compensation across an entire string set (or an ensemble of strings from a variety of manufacturers). In this case, \eqn{rms_def} becomes
 \begin{equation}\label{eqn:rms_def_m}
\overline{\Delta \nu}_\text{rms} \equiv \sqrt{\frac{\sum_{m = 1}^{m_\text{max}} \sum_{n = 1}^{n_\text{max}} \left[\Delta \nu^{(m)}_{n}\right]^2}{m_\text{max}\, n_\text{max}}}\, ,
 \end{equation}
where $m$ labels the strings in the set, and \eqn{error_tot} has been updated to become
 \begin{equation}\label{eqn:error_tot_m}
\Delta \nu^{(m)}_n \approx \frac{1200}{\ln(2)}\, \left\{ \left(\gamma_n - 1\right) \left[B_0^{(m)} - \frac{\Delta S}{X_0}\right] + \frac{\Delta N}{X_0} + \half\, \kappa^{(m)}\, Q_n \right\}\, .
 \end{equation}
If we rigorously follow the same approach that we used to arrive at \eqn{rms_sol}, in the multi-string case we obtain
 \begin{equation}\label{eqn:rms_sol_multi}
\begin{bmatrix}
  \Delta S \\
  \Delta N
\end{bmatrix} = \frac{1}{m_\text{max}}\, 
\begin{bmatrix}
  \sum_{m = 1}^{m_\text{max}} \Delta S^{(m)} \\
  \sum_{m = 1}^{m_\text{max}} \Delta N^{(m)}
\end{bmatrix}\, ,
 \end{equation}
where
 \begin{equation}\label{eqn:rms_sol_uni}
\begin{bmatrix}
  \Delta S^{(m)} \\
  \Delta N^{(m)}
\end{bmatrix} = \frac{X_0}{\sigma_1^2 - \sigma_0\, \sigma_2}\,
\begin{bmatrix}
  -\sigma_0 & \sigma_1 \\
  -\sigma_1 & \sigma_2
\end{bmatrix}\,
\begin{bmatrix}
  \sigma_2\, B_0^{(m)} + \half\, \kappa^{(m)}\, \overline{Q}_1 \\
  \sigma_1\, B_0^{(m)} + \half\, \kappa^{(m)}\, \overline{Q}_0
\end{bmatrix}\, .
 \end{equation}
In other words, we can find the optimum values for both $\Delta S$ and $\Delta N$ simply by averaging the corresponding setbacks over a commercially interesting collection of string sets. 
 %%%%%%%%%%%%%%%%%%%%%%%%%%%%%%%%%%%%%%%%%%%%%%%%%%%%%%%%%%%%%%%%%%%%%%%%%%%%%%
%
% Appendix file included in main project file using \input{}
%
% Assumes that LaTeX2e macros and packages defined in cg_comp.sty are
%   available
%
%%%%%%%%%%%%%%%%%%%%%%%%%%%%%%%%%%%%%%%%%%%%%%%%%%%%%%%%%%%%%%%%%%%%%%%%%%%%%%

 \section{Other Classical Guitar String Sets\label{app:specs}}
 
 \subsection{Specifications}

\begin{table}[htbp]
  \centering
  \caption{\label{tbl:ej43_ips} String properties for the D'Addario Pro-Arte Nylon Classical Guitar Strings -- Light Tension (EJ43). The corresponding scale length is 25.5~inches.}
    \begin{tabular}{lcccc}
    \hline \hline
    String  & Note  & \multicolumn{1}{l}{Diameter (in)} & \multicolumn{1}{l}{Density (lb/in)} & \multicolumn{1}{l}{Tension (lb)} \\
    \hline
    J4301 & E$_4$  & 0.0275 & $2.024 \times 10^{-5}$ & 14.8 \\
    J4302 & B$_3$  & 0.0317 & $2.729 \times 10^{-5}$ & 11.2 \\
    J4303 & G$_3$  & 0.0397 & $4.525 \times 10^{-5}$ & 11.7 \\
    J4304 & D$_3$  & 0.0280 & $1.020 \times 10^{-4}$ & 14.8 \\
    J4305 & A$_2$  & 0.0330 & $1.535 \times 10^{-4}$ & 12.5 \\
    J4306 & E$_2$  & 0.0420 & $2.888 \times 10^{-4}$ & 13.2 \\
    \hline
    \end{tabular}%
  \label{tab:addlabel}%
\end{table}%

\begin{table}[htbp]
  \centering
  \caption{\label{tbl:ej43_mks} String properties for the D'Addario Pro-Arte Nylon Classical Guitar Strings -- Light Tension (EJ43). The corresponding scale length is 650~mm.}
    \begin{tabular}{lcccc}
    \hline \hline
    String  & Note  & \multicolumn{1}{l}{Radius (mm)} & \multicolumn{1}{l}{Density (kg/mm)} & \multicolumn{1}{l}{Tension (N)} \\
    \hline
    J4301 & E$_4$  & 0.349 & $3.615 \times 10^{-7}$ & 66.4 \\
    J4302 & B$_3$  & 0.403 & $4.875 \times 10^{-7}$ & 50.2 \\
    J4303 & G$_3$  & 0.504 & $8.083 \times 10^{-7}$ & 52.5 \\
    J4304 & D$_3$  & 0.356 & $1.823 \times 10^{-6}$ & 66.4 \\
    J4305 & A$_2$  & 0.419 & $2.741 \times 10^{-6}$ & 56.1 \\
    J4306 & E$_2$  & 0.533 & $5.159 \times 10^{-6}$ & 59.2 \\
    \hline
    \end{tabular}%
  \label{tab:addlabel}%
\end{table}%

\begin{table}[htbp]
  \centering
  \caption{\label{tbl:ej46_ips} String properties for the D'Addario Pro-Arte Nylon Classical Guitar Strings -- Hard Tension (EJ46). The corresponding scale length is 25.5~inches.}
    \begin{tabular}{lcccc}
    \hline \hline
    String  & Note  & \multicolumn{1}{l}{Diameter (in)} & \multicolumn{1}{l}{Density (lb/in)} & \multicolumn{1}{l}{Tension (lb)} \\
    \hline
    J4601 & E$_4$  & 0.0285 & $2.161 \times 10^{-5}$ & 15.8 \\
    J4602 & B$_3$  & 0.0327 & $2.924 \times 10^{-5}$ & 12.0 \\
    J4603 & G$_3$  & 0.0410 & $4.795 \times 10^{-5}$ & 12.4 \\
    J4604 & D$_3$  & 0.0300 & $1.124 \times 10^{-4}$ & 16.3 \\
    J4605 & A$_2$  & 0.0360 & $1.952 \times 10^{-4}$ & 15.9 \\
    J4606 & E$_2$  & 0.0440 & $3.173 \times 10^{-4}$ & 14.5 \\
    \hline
    \end{tabular}%
  \label{tab:addlabel}%
\end{table}%

\begin{table}[htbp]
  \centering
  \caption{\label{tbl:ej46_mks} String properties for the D'Addario Pro-Arte Nylon Classical Guitar Strings -- Hard Tension (EJ46). The corresponding scale length is 650~mm.}
    \begin{tabular}{lcccc}
    \hline \hline
    String  & Note  & \multicolumn{1}{l}{Radius (mm)} & \multicolumn{1}{l}{Density (kg/mm)} & \multicolumn{1}{l}{Tension (N)} \\
    \hline
    J4601 & E$_4$  & 0.362 & $3.860 \times 10^{-7}$ & 70.9 \\
    J4602 & B$_3$  & 0.415 & $5.223 \times 10^{-7}$ & 53.8 \\
    J4603 & G$_3$  & 0.521 & $8.565 \times 10^{-7}$ & 55.6 \\
    J4604 & D$_3$  & 0.381 & $2.007 \times 10^{-6}$ & 73.1 \\
    J4605 & A$_2$  & 0.457 & $3.487 \times 10^{-6}$ & 71.3 \\
    J4606 & E$_2$  & 0.559 & $5.667 \times 10^{-6}$ & 65.0 \\
    \hline
    \end{tabular}%
  \label{tab:addlabel}%
\end{table}%

\begin{table}[htbp]
  \centering
  \caption{\label{tbl:ej44_ips} String properties for the D'Addario Pro-Arte Nylon Classical Guitar Strings -- Extra Hard Tension (EJ44). The corresponding scale length is 25.5~inches.}
    \begin{tabular}{lcccc}
    \hline \hline
    String  & Note  & \multicolumn{1}{l}{Diameter (in)} & \multicolumn{1}{l}{Density (lb/in)} & \multicolumn{1}{l}{Tension (lb)} \\
    \hline
    J4401 & E$_4$  & 0.0290 & $2.243 \times 10^{-5}$ & 16.4 \\
    J4402 & B$_3$  & 0.0333 & $3.046 \times 10^{-5}$ & 12.5 \\
    J4403 & G$_3$  & 0.0416 & $4.989 \times 10^{-5}$ & 12.9 \\
    J4404 & D$_3$  & 0.0300 & $1.124 \times 10^{-4}$ & 16.3 \\
    J4405 & A$_2$  & 0.0360 & $1.952 \times 10^{-4}$ & 15.9 \\
    J4406 & E$_2$  & 0.0450 & $3.435 \times 10^{-4}$ & 15.7 \\
    \hline
    \end{tabular}%
  \label{tab:addlabel}%
\end{table}%

\begin{table}[htbp]
  \centering
  \caption{\label{tbl:ej44_mks} String properties for the D'Addario Pro-Arte Nylon Classical Guitar Strings -- Hard Tension (EJ44). The corresponding scale length is 650~mm.}
    \begin{tabular}{lcccc}
    \hline \hline
    String  & Note  & \multicolumn{1}{l}{Radius (mm)} & \multicolumn{1}{l}{Density (kg/mm)} & \multicolumn{1}{l}{Tension (N)} \\
    \hline
    J4401 & E$_4$  & 0.368 & $4.007 \times 10^{-7}$ & 73.6 \\
    J4402 & B$_3$  & 0.423 & $5.441 \times 10^{-7}$ & 56.1 \\
    J4403 & G$_3$  & 0.528 & $8.912 \times 10^{-7}$ & 57.9 \\
    J4404 & D$_3$  & 0.381 & $2.007 \times 10^{-6}$ & 73.1 \\
    J4405 & A$_2$  & 0.457 & $3.487 \times 10^{-6}$ & 71.3 \\
    J4406 & E$_2$  & 0.571 & $6.136 \times 10^{-6}$ & 70.4 \\
    \hline
    \end{tabular}%
  \label{tab:addlabel}%
\end{table}%
 
 \subsection{Derived Properties}
\begin{table}%[htbp]
  \centering
  \caption{\label{tbl:ej43_props} Derived physical properties of the D'Addario Pro-Arte Nylon Classical Guitar Strings -- Light Tension (EJ43). The corresponding scale length is 650 mm.}
    \begin{tabular}{lcccc}
    \hline \hline
    String  & $R$ & $\kappa$ & Modulus (GPa) & Stiffness \\
    \hline
    J4301 & 2.67 $\times 10^{4}$ & 29.8 & 5.17 & 2.94 $\times 10^{-3}$ \\
    J4302 & 1.09 $\times 10^{4}$ & 11.6 & 1.14 & 2.11 $\times 10^{-3}$ \\
    J4303 & 6.63 $\times 10^{4}$ & 75.6 & 4.97 & 6.75 $\times 10^{-3}$ \\
    J4304 & 2.60 $\times 10^{4}$ & 29.1 & 4.86 & 2.95 $\times 10^{-3}$ \\
    J4305 & 2.66 $\times 10^{4}$ & 29.7 & 3.02 & 3.52 $\times 10^{-3}$ \\
    J4306 & 1.40 $\times 10^{4}$ & 15.2 & 1.01 & 3.20 $\times 10^{-3}$ \\
    \hline
    \end{tabular}%
  \label{tab:addlabel}%
\end{table}%

\begin{table}%[htbp]
  \centering
  \caption{\label{tbl:ej46_props} Derived physical properties of the D'Addario Pro-Arte Nylon Classical Guitar Strings -- Hard Tension (EJ46). The corresponding scale length is 650 mm.}
    \begin{tabular}{lcccc}
    \hline \hline
    String  & $R$ & $\kappa$ & Modulus (GPa) & Stiffness \\
    \hline
    J4601 & 10.2 $\times 10^{4}$ & 116.5 & 20.1 & 6.01 $\times 10^{-3}$ \\
    J4602 & 4.13 $\times 10^{4}$ & 46.7 & 4.65 & 4.37 $\times 10^{-3}$ \\
    J4603 & 2.96 $\times 10^{4}$ & 33.2 & 2.16 & 4.61 $\times 10^{-3}$ \\
    J4604 & 4.66 $\times 10^{4}$ & 52.8 & 8.47 & 4.26 $\times 10^{-3}$ \\
    J4605 & 1.79 $\times 10^{4}$ & 19.7 & 2.13 & 3.12 $\times 10^{-3}$ \\
    J4606 & 2.65 $\times 10^{4}$ & 29.6 & 1.96 & 4.68 $\times 10^{-3}$ \\
    \hline
    \end{tabular}%
  \label{tab:addlabel}%
\end{table}%

\begin{table}%[htbp]
  \centering
  \caption{\label{tbl:ej44_props} Derived physical properties of the D'Addario Pro-Arte Nylon Classical Guitar Strings -- Extra Hard Tension (EJ44). The corresponding scale length is 650 mm.}
    \begin{tabular}{lcccc}
    \hline \hline
    String  & $R$ & $\kappa$ & Modulus (GPa) & Stiffness \\
    \hline
    J4401 & 2.47 $\times 10^{4}$ & 27.6 & 20.1 & 2.98 $\times 10^{-3}$ \\
    J4402 & 5.64 $\times 10^{4}$ & 64.2 & 4.65 & 5.21 $\times 10^{-3}$ \\
    J4403 & 6.21 $\times 10^{4}$ & 70.7 & 2.16 & 6.83 $\times 10^{-3}$ \\
    J4404 & 4.66 $\times 10^{4}$ & 52.8 & 8.47 & 4.26 $\times 10^{-3}$ \\
    J4405 & 1.79 $\times 10^{4}$ & 19.7 & 2.13 & 3.12 $\times 10^{-3}$ \\
    J4406 & 2.56 $\times 10^{4}$ & 28.5 & 1.96 & 4.70 $\times 10^{-3}$ \\
    \hline
    \end{tabular}%
  \label{tab:addlabel}%
\end{table}%

 \subsection{Predicted Setbacks}
\begin{table}%[htbp]
  \centering
  \caption{\label{tbl:ej43_setbacks} Predicted setbacks for the D'Addario Pro-Arte Nylon Classical Guitar Strings -- Light Tension (EJ43) on the Alhambra 8P classical guitar.}
    \begin{tabular}{lcccc}
    \hline \hline
    String  & $\Delta S$~(mm) & $\Delta N$~(mm) \\
    \hline
    J4301 & 1.91 & -0.31 \\
    J4302 & 1.37 & -0.12 \\
    J4303 & 4.38 & -0.79 \\
    J4304 & 1.92 & -0.30 \\
    J4305 & 2.29 & -0.31 \\
    J4306 & 2.08 & -0.16 \\
    \hline \hline
    Mean & 2.32 & -0.33 \\
    \hline
    \end{tabular}%
  \label{tab:addlabel}%
\end{table}%

\begin{table}%[htbp]
  \centering
  \caption{\label{tbl:ej46_setbacks} Predicted setbacks for the D'Addario Pro-Arte Nylon Classical Guitar Strings -- Hard Tension (EJ46) on the Alhambra 8P classical guitar.}
    \begin{tabular}{lcccc}
    \hline \hline
    String  & $\Delta S$~(mm) & $\Delta N$~(mm) \\
    \hline
    J4601 & 3.91 & -1.20 \\
    J4602 & 2.84 & -0.49 \\
    J4603 & 3.00 & -0.34 \\
    J4604 & 2.77 & -0.55 \\
    J4605 & 2.03 & -0.20 \\
    J4606 & 3.04 & -0.31 \\
    \hline \hline
    Mean & 2.93 & -0.52 \\
    \hline
    \end{tabular}%
  \label{tab:addlabel}%
\end{table}%

\begin{table}%[htbp]
  \centering
  \caption{\label{tbl:ej44_setbacks} Predicted setbacks for the D'Addario Pro-Arte Nylon Classical Guitar Strings -- Extra Hard Tension (EJ44) on the Alhambra 8P classical guitar.}
    \begin{tabular}{lcccc}
    \hline \hline
    String  & $\Delta S$~(mm) & $\Delta N$~(mm) \\
    \hline
    J4401 & 1.93 & -0.29 \\
    J4402 & 3.39 & -0.67 \\
    J4403 & 4.44 & -0.73 \\
    J4404 & 2.77 & -0.55 \\
    J4405 & 2.03 & -0.20 \\
    J4406 & 3.05 & -0.30 \\
    \hline \hline
    Mean & 2.94 & -0.46 \\
    \hline
    \end{tabular}%
  \label{tab:addlabel}%
\end{table}%

 \subsection{Compensation Plots}
 
  \begin{figure}
  \centering
  \begin{subfigure}[b]{0.45\textwidth}
   \centering
   \includegraphics[width=3.25in]{figures/shift_alhambra8p_ej43_null}
   \caption{Uncompensated}
   \label{fig:shift_alhambra8p_ej43_null}
  \end{subfigure}
  \hspace{0.25in}
  \begin{subfigure}[b]{0.45\textwidth}
   \centering
   \includegraphics[width=3.25in]{figures/shift_alhambra8p_ej43_factory}
   \caption{Factory guitar}
   \label{fig:shift_alhambra8p_ej43_factory}
  \end{subfigure}
  \par\vspace{0.25in}
  \begin{subfigure}[b]{0.45\textwidth}
   \centering
   \includegraphics[width=3.25in]{figures/shift_alhambra8p_ej43_mean}
   \caption{Mean compensation}
   \label{fig:shift_alhambra8p_ej43_mean}
  \end{subfigure}
  \hspace{0.25in}
  \begin{subfigure}[b]{0.45\textwidth}
   \centering
   \includegraphics[width=3.25in]{figures/shift_alhambra8p_ej43_full}
   \caption{Full compensation}
   \label{fig:shift_alhambra8p_ej43_full}
  \end{subfigure}
  \caption{\label{fig:compensation} Frequency shift (in cents) for an Alhambra 8P guitar with D'Addario Pro-Arte Nylon Classical Guitar Strings -- Light Tension (EJ43). Four different strategies of saddle and nut compensation are illustrated.}
 \end{figure}




 \begin{figure}
  \centering
  \begin{subfigure}[b]{0.45\textwidth}
   \centering
   \includegraphics[width=3.25in]{figures/shift_alhambra8p_ej46_null}
   \caption{Uncompensated}
   \label{fig:shift_alhambra8p_ej46_null}
  \end{subfigure}
  \hspace{0.25in}
  \begin{subfigure}[b]{0.45\textwidth}
   \centering
   \includegraphics[width=3.25in]{figures/shift_alhambra8p_ej46_factory}
   \caption{Factory guitar}
   \label{fig:shift_alhambra8p_ej46_factory}
  \end{subfigure}
  \par\vspace{0.25in}
  \begin{subfigure}[b]{0.45\textwidth}
   \centering
   \includegraphics[width=3.25in]{figures/shift_alhambra8p_ej46_mean}
   \caption{Mean compensation}
   \label{fig:shift_alhambra8p_ej46_mean}
  \end{subfigure}
  \hspace{0.25in}
  \begin{subfigure}[b]{0.45\textwidth}
   \centering
   \includegraphics[width=3.25in]{figures/shift_alhambra8p_ej46_full}
   \caption{Full compensation}
   \label{fig:shift_alhambra8p_ej46_full}
  \end{subfigure}
  \caption{\label{fig:compensation} Frequency shift (in cents) for an Alhambra 8P guitar with D'Addario Pro-Arte Nylon Classical Guitar Strings -- Hard Tension (EJ46). Four different strategies of saddle and nut compensation are illustrated.}
 \end{figure}

 \begin{figure}
  \centering
  \begin{subfigure}[b]{0.45\textwidth}
   \centering
   \includegraphics[width=3.25in]{figures/shift_alhambra8p_ej44_null}
   \caption{Uncompensated}
   \label{fig:shift_alhambra8p_ej44_null}
  \end{subfigure}
  \hspace{0.25in}
  \begin{subfigure}[b]{0.45\textwidth}
   \centering
   \includegraphics[width=3.25in]{figures/shift_alhambra8p_ej44_factory}
   \caption{Factory guitar}
   \label{fig:shift_alhambra8p_ej44_factory}
  \end{subfigure}
  \par\vspace{0.25in}
  \begin{subfigure}[b]{0.45\textwidth}
   \centering
   \includegraphics[width=3.25in]{figures/shift_alhambra8p_ej44_mean}
   \caption{Mean compensation}
   \label{fig:shift_alhambra8p_ej44_mean}
  \end{subfigure}
  \hspace{0.25in}
  \begin{subfigure}[b]{0.45\textwidth}
   \centering
   \includegraphics[width=3.25in]{figures/shift_alhambra8p_ej44_full}
   \caption{Full compensation}
   \label{fig:shift_alhambra8p_ej44_full}
  \end{subfigure}
  \caption{\label{fig:compensation} Frequency shift (in cents) for an Alhambra 8P guitar with D'Addario Pro-Arte Nylon Classical Guitar Strings -- Extra Hard Tension (EJ44). Four different strategies of saddle and nut compensation are illustrated.}
 \end{figure}





 \addtotoc{\refname}
 \bibliographystyle{rgb}
 \bibliography{music,cgic}

 \end{document}
