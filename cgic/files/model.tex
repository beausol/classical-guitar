%%%%%%%%%%%%%%%%%%%%%%%%%%%%%%%%%%%%%%%%%%%%%%%%%%%%%%%%%%%%%%%%%%%%%%%%%%%%%%
%
% Section file included in main project file using \input{}
%
% Assumes that LaTeX2e macros and packages defined in cg_comp.sty are
%   available
%
%%%%%%%%%%%%%%%%%%%%%%%%%%%%%%%%%%%%%%%%%%%%%%%%%%%%%%%%%%%%%%%%%%%%%%%%%%%%%%

 \section{Simple Model of Guitar Intonation\label{sct:model}}
The starting point for prior efforts to understand guitar intonation and compensation~\cite{ref:byers1996cgi,ref:varieschi2010icf} is a formula for $f_{q}$, the transverse vibration frequency harmonic $q$ of a stiff string, originally published by Morse in 1936~\cite{ref:morse1981vsb,ref:fletcher1964nvf, ref:fletcher2005pmb}:
 \begin{equation}\label{eqn:f_m_clamped}
f_{q} = \frac{q}{2\, L}\, \sqrt{\frac{T}{\mu}} \left[ 1 + 2 B + 4 \left(1 + \frac{\pi^2\, q^2}{8}\right) B^2 \right]\, .
 \end{equation}
Here $L$ is the length of the string, $T$ and $\mu$ are its tension and linear mass density, respectively, and $B$ is a small ``bending stiffness'' coefficient to capture the relevant mechanical properties of the string. For a homogeneous string with a cylindrical cross-section, $B$ is given by
 \begin{equation} \label{eqn:b_def}
  B \equiv \sqrt{\frac{E\, \mathcal{A}\, s^2}{L^2 T}}\, ,
 \end{equation}
where $\mathcal{A}$ and $s$ are the cross-sectional area and the radius of gyration of the string, respectively, and $E$ is Young's modulus (or the modulus of elasticity). But it's unlikely that \eqn{f_m_clamped} accurately describes the resonant frequencies of a nylon string on a classical guitar. First, it is derived by assuming that the vibration of the string is polarized vertically (perpendicular to the plane of the guitar top). This is true for a piano string, but not for a classical guitar string, which is polarized elliptically with the major axis parallel to the guitar top. Second, the factor of two in front of the bending stiffness arises from the assumption that the string is ``clamped'' at both ends, so that a particular set of symmetric mathematical boundary conditions must be applied to the partial differential equation (PDE) describing transverse vibrations of the string. However, measurements of the frequency of a stiff piano string showed that neither symmetric clamped nor ``pinned'' boundary conditions were completely correct~\cite{ref:fletcher1964nvf}. In addition, \eqn{f_m_clamped} predicts values of saddle setbacks that are about twice as large as those used by experienced luthiers based on trial and error~\cite{ref:buckland2021ror}.

As a compromise, we assume that the string is clamped at the nut but pinned at the saddle, and we neglect the impact of the polarization of the vibrating string. In \app{freq}, we solve the PDE using these non-symmetric boundary conditions, and find
 \begin{equation} \label{eqn:f_m_stiff}
f_q = \frac{q}{2\, L}\, \sqrt{\frac{T}{\mu}} \left[ 1 + B + \left( 1 + \half q^2 \pi^2 \right) B^2 \right]\, .
 \end{equation}
Note that this expression is valid only when $B \ll 1$. We'll see that a typical nylon guitar string has $B \approx 2 - 3 \times 10^{-3}$. In this case, the quadratic $B$ term in \eqn{f_m_stiff} is only 2\% as large as the linear term, and can generally be neglected. (We will include it in our numerical computations for completeness.) We should use \eqn{f_m_stiff} with some caution, because the chemistry, materials science, and physics of nylon strings (particularly the wound bass strings) are quite complicated~\cite{ref:blanc1996nvb,ref:lynchaird2017mpn,ref:lynchaird2018cmp}. With this in mind, we check the validity of this equation for the nylon strings we measure in \sct{exp}.

 \begin{figure}
  \centering
  \includegraphics[width=7.0in]{figures/fretting_schematic}
  \caption{\label{fig:guitar_schematic} A simple (side-view) schematic of the classical guitar used in this model.}
 \end{figure}

Our model is based on the schematic of the guitar shown in \fig{guitar_schematic}. The scale length of the guitar is $X_0$, but we allow the inside edges of both the saddle and the nut to be set back an additional distance $\Delta S$ and $\Delta N$, respectively. The location on the $x$-axis of the center of the $n^\textrm{th}$ fret is $X_n$. In the $y$ direction, $y = 0$ is taken as the surface of the fingerboard; the height of each fret is $a$, the height of the nut (i.e., the distance between the fingerboard and the \emph{bottom} of the string) is $a + b$, and the height of the saddle is $a + b + c$. (For the moment, we are neglecting the art of \emph{relief} practiced by expert luthiers that adjusts the value of $b$ up the fretboard and strings. We discuss this effect below in \sct{model_tension}.) $L_n$ is the \emph{resonant length} of the string from the saddle to the center of fret $n$, and $L^\prime_n$ is the length of the string from the fret to the nut. The total length of the string is defined as $\mathcal{L}_n \equiv L_n + L^\prime_n$. As discussed in more detail in \app{fret}, we have chosen to include a line-segment intersection at a distance $d$ behind fret $n$ to represent the slight increase in the distance $L_n^\prime$ caused by a finger. This differs from previous studies of guitar intonation and compensation~\cite{ref:byersgal,ref:byers1996cgi,ref:varieschi2010icf}, but our approach is consistent with our empirical observations for nylon strings.

We start with the form of the fundamental frequency of a fretted string given by \eqn{f_m_stiff} with $q = 1$, and apply it to the frequency of a string pressed just behind the $n^\mathrm{th}$ fret:
 \begin{equation} \label{eqn:f_n_def}
f_n = \frac{1}{2\, L_n}\, \sqrt{\frac{T_n}{\mu_n}} \left[ 1 + B_n + \left(1 + \frac{\pi^2}{2}\right) B_n^2 \right]\, ,
 \end{equation}
where $T_n$ and $\mu_n$ are the modified tension and the linear mass density of the fretted string, and
 \begin{equation} \label{eqn:b_n_def}
B_n \equiv \sqrt{\frac{E\, \mathcal{A}\, s^2}{4\, T_n\, L_n^2}}\, .
 \end{equation}
We note that $T_n$ and $\mu_n$ depend on $\mathcal{L}_n$, the \emph{total} length of the fretted string from the saddle to the nut. Ideally, in the 12-TET system~\cite{ref:durfee2015pms},
 \begin{equation} \label{eqn:f_n_tet}
f_n = \gamma_n\, f_0\, , \qquad \textrm{(12-TET~ideal)}
 \end{equation}
where $f_0$ is the frequency of the open (unfretted) string, and
 \begin{equation} \label{eqn:gamme_n_def}
\gamma_n \equiv 2^{n / 12}\, .
 \end{equation}
Therefore, the error interval --- the difference between the fundamental frequency of the fretted string and the corresponding perfect 12-TET frequency --- expressed in cents is given by
 \begin{equation}\label{eqn:error_def}
 \begin{split}
\Delta \nu_n &= 1200\, \log_2\left( \frac{f_n}{\gamma_n\, f_0} \right) \\
%&= 1200\, \log_2 \left( \frac{L_0}{\gamma_n\, L_n}\, \sqrt{\frac{\mu_0}{\mu_n}\, \frac{T_n}{T_0}}\, \frac{1 + B_n}{1 + B_0} \right) \\
&= 1200\, \log_2 \left( \frac{L_0}{\gamma_n\, L_n} \right) + 600\, \log_2 \left(  \frac{\mu_0}{\mu_n} \right) + 600\, \log_2 \left( \frac{T_n}{T_0} \right) \\
&\qquad + 1200\, \log_2 \left[ \frac{1 + B_n + (1 + \pi^2/2)\, B_n^2}{1 + B_0 + (1 + \pi^2/2)\, B_0^2} \right]\, ,
 \end{split}
 \end{equation}
where $\log_2$ is the (binary) logarithm function calculated with base $2$.

The final form of \eqn{error_def} makes it clear that --- for nylon guitar strings --- there are four contributions to intonation:
 \begin{enumerate}
  \item
   \emph{Resonant Length}: The first term represents the error caused by the increase in the length of the fretted string $L_n$ compared to the ideal length $X_n$, which would be obtained if $b = c = d = 0$ and $\Delta S = \Delta N = 0$.
  \item
   \emph{Linear Mass Density}: The second term is the error caused by the reduction of the linear mass density of the fretted string. This effect will depend on the \emph{total} length of the string $\mathcal{L}_n = L_n + L^\prime_n$.
  \item
   \emph{Tension}: The third term is the error caused by the \emph{increase} of the tension in the string arising from the stress and strain applied to the string by fretting. This effect will also depend on the total length of the string $\mathcal{L}_n$.
  \item
   \emph{Bending Stiffness}: The fourth and final term is the error caused by the change in the bending stiffness coefficient arising from the decrease in the vibrating length of the string from $L_0$ to $L_n$.
 \end{enumerate}
Note that the properties of the logarithm function have \emph{decoupled} these physical effects by converting multiplication into addition. We will discuss each of these sources of error in turn below.

In the discussion that follows, we'll test our approximations for a prototypical Classical Guitar with the specifications listed in \tbl{mcg_specs}. Refer to \fig{guitar_schematic} for a graphical representation of these parameters. In addition, as we develop models of the physical effects discussed above, we'll assume that the guitar string has the properties listed in \tbl{string_specs}. The string constant $\kappa$ and the open-string bending stiffness $B_0$ are introduced in \sct{model_tension} and \sct{model_stiffness} respectively. The linear frequency shift parameter $R$ is discussed in \sct{tot_freq_shift}, and a method for determining both $\kappa$ and $B_0$ in terms of $R$ is discussed.

\begin{table}%[htbp]
  \centering
  \caption{\label{tbl:mcg_specs} Default specifications for the prototypical Classical Guitar modeled in this section. The default values of $d$, $\Delta S$, and $\Delta N$ can be either zero or the nonzero value listed in the table and discussed in the text.}
  \begin{tabular}{cll}
    \toprule
    Parameter & Description & Default Value (mm) \\
    \midrule
    $X_0$ & Scale length& $650$ \\
    $b$ & Height of the nut above fret $1$& $1$ \\
    $c$ & Height of the saddle above the nut & $4$ \\
    $d$ & Fretting distance & $0$ or $10$ \\
    $\Delta S$ & Saddle setback & $0$ or $1.8$ \\
    $\Delta N$ & Nut setback & $0$ or $-0.38$ \\
    \bottomrule
  \end{tabular}
\end{table}%

\begin{table}%[htbp]
  \centering
  \caption{\label{tbl:string_specs} Default specifications for a prototypical guitar string. The string constant $\kappa$ and the open-string bending stiffness $B_0$ are introduced in \sct{model_tension} and \sct{model_stiffness} respectively, and the linear frequency shift parameter $R$ is discussed in \sct{tot_freq_shift}.}
  \begin{tabular}{cll}
    \toprule
    Parameter & Description & Default Value \\
    \midrule
    $\rho$ & String radius in mm & $0.43$ \\
    $R$ & Linear frequency shift parameter & $25$ \\
    $\kappa$ & String tension constant & $51$ \\
    $B_0$ & Open-string bending stiffness & $0.00236$ \\
    \bottomrule
  \end{tabular}
\end{table}%

 \subsection{Resonant Length}
The length $L_0$ of the open (unfretted) guitar string can be calculated quickly by referring to \fig{guitar_schematic}. We find:
 \begin{equation}  \label{eqn:l_0_def}
L_0 = \sqrt{\left(X_0 + \Delta S + \Delta N\right)^2 + c^2} \approx X_0 + \Delta S + \Delta N + \frac{c^2}{2\, X_0}\, ,
 \end{equation}
where the approximation arising from the Taylor series is valid to second order in all small distances since $\{\Delta S, \Delta N, b, c\} \ll X_0^2$. Similarly, the resonant length $L_n$ is given by
 \begin{equation}  \label{eqn:l_n_def}
L_n = \sqrt{\left(X_n + \Delta S\right)^2 + (b + c)^2} \approx X_n + \Delta S + \frac{(b + c)^2}{2\, X_n}\, .
 \end{equation}
Then --- if the guitar has been manufactured such that $X_n = X_0 / \gamma_n$ --- the resonant length error determined by the first term in the last line of \eqn{error_def} is approximately
%  \begin{equation}
%  1200\, \log_2 \left( \frac{L_0}{\gamma_n\, L_n} \right) \approx -\frac{1200}{\ln(2)} \left[ \frac{\left(\gamma_n - 1\right) \Delta S - \Delta N}{X_0} + \frac{\gamma_n^2 (b + c)^2 - c^2}{2\, X_0^2}\right]
%  \end{equation}
\begin{equation} \label{eqn:rle_taylor}
  \begin{split}
    1200\, \log_2 \left( \frac{L_0}{\gamma_n\, L_n} \right) &\approx \frac{1200}{\ln(2)} \left[ \frac{\Delta N - \left(\gamma_n - 1\right) \Delta S}{X_0} - \frac{(\Delta N + \Delta S)^2 - \gamma_n^2\, \Delta S^2}{2\, X_0^2}\right. \\ &\qquad\qquad - \left. \frac{\gamma_n^2 (b + c)^2 - c^2}{2\, X_0^2}\right]\, .
  \end{split}
\end{equation}
If the guitar is uncompensated, so that $\Delta S = \Delta N = 0$, the magnitude of this error on our Classical Guitar can be neglected in approximate treatments. However, we'll see that choosing $\Delta S > 0$ and $\Delta N < 0$ will allow us to substantially compensate for frequency shift contributions from other effects. In this case, we note that the three terms inside the bracket on the \rhs of \eqn{rle_taylor} are $\{-3.5 \times 10^{-3}, 1.4 \times 10^{-5}, -1.0 \times 10^{-4}\}$, respectively, for the parameter values given by \tbl{mcg_specs}, corresponding to frequency shifts of $\{-6.05, 0.02, -0.17\}$~cents. Therefore, the second two terms are negligible compared to the first, and we can approximate the resonant length error --- for the purposes of estimating setbacks in \sct{comp} --- as
\begin{equation} \label{eqn:rle_approx}
  1200\, \log_2 \left( \frac{L_0}{\gamma_n\, L_n} \right) \approx \frac{1200}{\ln(2)} \left[ \frac{\Delta N - \left(\gamma_n - 1\right) \Delta S}{X_0}\right]\, .
\end{equation}
We'll include the term in \eqn{rle_taylor} that is quadratic in $b$ and $c$ in our computation of setbacks detailed in \app{rms}, and we'll use \eqn{l_0_def} and \eqn{l_n_def} when computing frequency errors.

 \subsection{Linear Mass Density\label{sct:model_lmd}}
As discussed above, the linear mass density $\mu_0$ of an open (unfretted) string is simply the total mass $M$ of the string clamped between the saddle and the nut divided by the length $L_0$. Similarly, the mass density $\mu_n$ of a string held onto fret $N$ is $M/\mathcal{L}_n$. Therefore
 \begin{equation}
\frac{\mu_0}{\mu_n} = \frac{\mathcal{L}_n}{L_0} \equiv 1 + Q_n\, ,
 \end{equation}
where we have followed Byers and defined the normalized relative displacement~\cite{ref:byersgal,ref:byers1996cgi,ref:varieschi2010icf}
 \begin{equation} \label{eqn:q_n_def}
Q_n \equiv \frac{\mathcal{L}_n - L_0}{L_0}\, ,
 \end{equation}
where $\mathcal{L}_n = L_n + L_n^\prime$. After judicious use of similar triangles and the Pythagorean Theorem we calculate $L^\prime_n$ for $n \ge 1$ as
\begin{equation} \label{eqn:l_p_def}
  L^\prime_n = \frac{L_n}{X_n + \Delta S}\, d + \sqrt{\left(X_0 - X_n + \Delta N - d\right)^2 + \left(b + \frac{b + c}{X_n + \Delta S}\, d\right)^2}\, .
\end{equation}
When $d \ll X_0$, we can expand $Q_n$ to third order in all small distances and find
\begin{equation} \label{eqn:q_n_approx}
  Q_n \approx \frac{\left[ \gamma_n\, b + (\gamma_n - 1)\, c \right]^2}{2\, (\gamma_n - 1)\, X_0^2} \left( 1 + \frac{\gamma_n^2}{\gamma_n - 1}\, \frac{d}{X_0} \right)\, ,
\end{equation}
Although it is arguable whether this approximation is simpler than the exact expression given by \eqn{q_n_def}, it is quite clear that $Q_n$ does not depend significantly on the setbacks $\Delta S$ or $\Delta N$. For a guitar with the specifications listed in \tbl{mcg_specs}, $Q_n$ falls in the range $25 - 45 \times 10^{-6}$ for $d \le 10$~mm, corresponding to a net stretch of the string less than $0.03$~mm. For the same parameters, when $d = 10$~mm, we find that $\Delta \nu_{1} \approx 0.04$~cents, and is smaller at all other frets. Therefore, in general the shift due to linear mass density can be neglected without significant loss of accuracy in the approximate setback solutions we derive in \sct{comp}.

% In this case, we won't bother with a Taylor series expansion that includes $d \ne 0$, because it isn't significantly more illuminating than \eqn{l_p_def}.

% Since we expect that $Q_n \ll 1$, we can approximate the second term in the final line of \eqn{error_def} as
%  \begin{equation} \label{eqn:lmd_error}
% 600\, \log_2 \left(  \frac{\mu_0}{\mu_n} \right) \approx \frac{600}{\ln(2)}\, Q_n\, .
%  \end{equation}


% Referring to \fig{guitar_schematic}, we see that $\mathcal{L}_n = L_n + L^\prime_n$, and we calculate $L^\prime_n$ for $n \ge 1$ as
%  \begin{equation}
% L^\prime_n = \sqrt{\left(X_0 - X_n + \Delta N\right)^2 + b^2} \approx X_0 - X_n + \Delta N + \frac{b^2}{2 \left(X_0 - X_n\right)}\, ,
%  \end{equation}

%However, when $d = 0$, $Q_n$ is approximately
% \begin{equation}
%     \begin{split}
%         L^\prime_n &= \frac{L_n}{X_n + \Delta S}\, d + \sqrt{\left(X_0 - X_n + \Delta N - d\right)^2 + \left(b + \frac{b + c}{X_n + \Delta S}\, d\right)^2} \\
%         &\approx X_0 - X_n + \Delta N + \frac{b^2}{2 \left(X_0 - X_n\right)} + \frac{\left[(b + c)\, X_0 - c\, X_n\right]^2}{2 (X_0 - X_n)^2 X_n^2}\, d
%     \end{split}
% \end{equation}
% to first order in $d/X_0$. Therefore
% accurate to second order in all small distances. This result suggests that the relative displacement does not depend significantly on the setbacks $\Delta S$ and $\Delta N$.

% In \fig{qn_test}, we plot a comparison between the exact expression for the normalized displacement $Q_n$ given by \eqn{q_n_def} with the approximate expression given by \eqn{q_n_approx} for two values of $d$. Here the guitar has the specifications listed in \tbl{mcg_specs}: the exact curves use $\Delta S = 1.9$~mm and $\Delta N = -0.37$~mm, and the approximate curves ignore the setbacks entirely. The slight difference between the exact and approximate values of $Q_n$ for $d = 10$~mm at the first fret can be eliminated if we include a term quadratic in $d$ in \eqn{q_n_approx}. As predicted above, we see that the dependence of $Q_n$ on the setback values is minimal. We have scaled the relative displacement by a factor of $10^6$ for convenience. For example, at the first fret with $d = 10$~mm, $Q_1 \approx 47 \times 10^{-6}$. This value corresponds to a net stretch of the string at the first fret of $Q_1\, L_0 = 0.03$~mm. Ignoring the setbacks at the $12^{\textrm{th}}$ fret increases $Q_{12}$ by 1\%, a completely negligible amount. 

% We can use this result and \eqn{q_n_approx} to determine the increase in the relative displacement $Q_n$; we find
%  % \begin{equation} %\label{eqn:delta_q_n_approx}
%  %     \Delta Q_n \approx \left(\frac{\gamma_n}{\gamma_n - 1}\right)^2 \frac{b^2\, d}{2\, X_0^3}\, .
%  % \end{equation}
%  \begin{equation} %\label{eqn:delta_q_n_approx}
%      \begin{split}
%          \Delta Q_n &\approx \frac{\gamma_n^2\, d}{2\, X_0^3} \left( \frac{\gamma_n}{\gamma_n - 1}\, b + c \right)^2 \\
%          &= Q_n\, \frac{\gamma_n^2}{\gamma_n - 1}\, \frac{d}{X_0}\, .
%      \end{split}
%  \end{equation}
% where the approximation applies when $b^2 \ll (X_0 - X_n)^2$. Therefore, using \eqn{l_n_def}, we have
%  \begin{equation}
% \mathcal{L}_n = L_n + L^\prime_n \approx X_0 + \Delta S + \Delta N + \frac{(b + c)^2}{2\, X_n} + \frac{b^2}{2 \left(X_0 - X_n\right)}\, ,
%  \end{equation}
% and
% \begin{equation}
%   \begin{split}
%     Q_n &\approx \frac{1}{2\, X_0} \left[ \frac{(b + c)^2}{X_n} + \frac{b^2}{X_0 - X_n} - \frac{c^2}{X_0} \right] \\
%     &= \frac{\gamma_n}{2\, X_0^2} \left[ (b + c)^2 + \frac{b^2}{\gamma_{n} - 1} - \frac{c^2}{\gamma_n} \right] \\
%     &= \frac{\gamma_n - 1}{2\, X_0^2} \left( \frac{\gamma_n}{\gamma_n - 1}\, b + c \right)^2 \\
%     &= \frac{1}{2\, (\gamma_n - 1)\, X_0^2} \left[ \gamma_n\, b + (\gamma_n - 1)\, c \right]^2 \, .
%  \end{split}
%  \end{equation}

%  \begin{equation} \label{eqn:delta_q_n_approx}
%   \begin{split}
%       \Delta Q_n &\approx \frac{\gamma_n^2\, d}{2\, X_0^3} \left( \frac{\gamma_n}{\gamma_n - 1}\, b + c \right)^2 \\
%       &= Q_n\, \frac{\gamma_n^2}{\gamma_n - 1}\, \frac{d}{X_0}\, .
%   \end{split}
% \end{equation}

% Although it is arguable whether the approximation given by \eqn{q_n_approx} is simpler than the exact expression given by \eqn{q_n_def}, it is quite clear from both \eqn{q_n_approx} and \fig{qn_test} that $Q_n$ \emph{does not depend significantly on the setbacks} $\Delta S$ or $\Delta N$. For the same parameters, when $d = 10$~mm, $\Delta \nu_{1} \approx 0.04$~cents, and is smaller at all other frets. Therefore, in general the shift due to linear mass density can be neglected without significant loss of accuracy.

%If we add this shift due to the linear mass density to the residual quadratic resonant length shift given by \eqn{rle_approx}, then we find the total error
% \begin{equation} \label{eqn:quad_shift}
%\Delta \nu_n = \frac{300}{\ln(2)}\, \frac{\gamma_n}{X_0^2} \left[ \frac{b^2}{\gamma_n - 1} + \frac{c^2}{\gamma_n} - (2 \gamma_n - 1) (b + c)^2 \right]\, .
% \end{equation}
%For the same parameters, $\Delta \nu_{12} = -0.11$~cents, and $|\Delta \nu_n| < |\Delta \nu_{12}|$ for $n < 12$.

% \begin{figure}
%   \centering
%   \includegraphics[width=5.0in]{figures/tn_test}
%   \caption{\label{fig:tn_test} Comparison of the exact expression for the frequency shift due to tension increases given by the \lhs of \eqn{tension_shift} with the approximate expression given by the right-hand side.}
% \end{figure}

 \subsection{Tension\label{sct:model_tension}}
%The third term in \eqn{error_def} provides us withe frequency shift of a string
Counterintuitively, nylon classical guitar strings~\cite{ref:blanc1996nvb,ref:lynchaird2017mpn,ref:lynchaird2018cmp} have very different physical properties than those of steel strings~\cite{ref:grimes2014stp,ref:kemp2017puw,ref:kemp2020ibg}, with completely different stress-strain curves. When fresh nylon strings are brought up to the required tension for the first time, they are stretched by a macroscopic distance $\Delta L$ that varies from 7~cm (for the first $E_4$ string) to 2~cm (for the sixth $E_2$ string). After only a few minutes, the string must be re-tensioned because it has experienced nonlinear viscoelastic relaxation and has gone flat by at least a half-step. (This stage of tensioning and strain is not well described by theories of nonlinear elasticity in soft materials~\cite{ref:mihai2017hcn}.) This process continues for several hours until the strings begin to ``settle'' and remain properly tuned for longer periods; after about 10~hours, they will respond at the correct frequencies for more than an hour provided that the temperature in the room doesn't change significantly~\cite{ref:blanc1996nvb,ref:lynchaird2017mpn}. A string removed from the guitar after this stage has been reached will not relax back to its original ``out-of-the-box'' length --- it has been permanently deformed. The frequency of most settled nylon strings string can be ``dropped'' one whole step and returned to the initial value, but attempts to increase tension further will reach a nonlinear stage where the frequency increases much less quickly and will often result in a broken string.

We will focus on the response of a settled string to a differential longitudinal strain, and neglect the transverse stress that causes insignificant changes in the radius of the string~\cite{ref:lynchaird2017mpn}. As we show in \sct{exp}, in the settled regime we can infer an infinitesimal change $\delta L$ of a nylon string with length $L$ will result in a linear change in the tension by an amount~\cite{ref:landau1986toe}
\begin{equation} \label{eqn:youngs_mod_def}
  \delta T = \mathcal{A}\, E_\mathrm{eff}\, \frac{\delta L}{L}\, ,
\end{equation}
where $E_\mathrm{eff}$ is an effective linear modulus of elasticity representing the ratio of the differential stress $\delta T / \mathcal{A}$ to the differential strain $\delta L / L$. Therefore, we write the change in tension of a string stretched by touching fret $n$ as
 \begin{equation} \label{eqn:delta_t_n}
\delta T_n = \mathcal{A}\, E_\mathrm{eff}\, Q_n\, ,
 \end{equation}
where $Q_n$ is the normalized infinitesimal displacement --- here acting as the differential strain --- defined by \eqn{q_n_def}. Note that we are using the length $L_0$ of the unfretted string between the saddle and the nut as our reference length. In the case of steel strings, it may be appropriate to include the length of the string between the nut and the tuning peg if the tension on either side of the nut is the same~\cite{ref:kemp2020ibg}. But a nylon string on a classical guitar emerges from the outside of the nut at a sharp angle, and is held so tightly within the nut that $\delta L$ of a settled string is twice as large between the nut and the tuners than that of the vibrating string.

Based on these considerations, we write the tension in a settled string clamped to fret $n$ as
 \begin{equation} \label{eqn:t_n_def}
T_n = T_0 + \delta T_n = T_0 \left( 1 + \kappa\, Q_n \right)\, ,
 \end{equation}
where we have defined the dimensionless linear ``string constant''
 \begin{equation}\label{eqn:kappa_def}
\kappa \equiv \frac{\mathcal{A} E_\mathrm{eff}}{T_0}\, .
 \end{equation}
The corresponding frequency shift due to the increase in tension caused by fretting is therefore given by the third term in the final line of \eqn{error_def} as
\begin{equation} \label{eqn:tension_shift}
  600\, \log_2 \left(  \frac{T_n}{T_0} \right) = 600\, \log_2 \left( 1 + \kappa\, Q_n \right)\, .
\end{equation}
If we assume that $\kappa\, Q_n \ll 1$, then we can approximate  this expression as
\begin{equation} \label{eqn:tension_approx}
  600\, \log_2 \left(  \frac{T_n}{T_0} \right) \approx \frac{600}{\ln(2)}\, \kappa\, Q_n\, ,
\end{equation}
where now $Q_n$ is given by \eqn{q_n_approx}. In this form, it is clear that this frequency shift is larger than that caused by the linear mass density by a factor of $\kappa$.

In \fig{tnu_test}, we plot a comparison between the exact and approximate expressions for the frequency error resulting from the tension increase given by \eqn{tension_shift} and \eqn{tension_approx} for two values of $d$. The normalized displacement $Q_n$ is computed using \eqn{q_n_def} in the exact curves and \eqn{q_n_approx} in the approximate curves. Here the guitar has the specifications listed in \tbl{mcg_specs}: the exact curves use $\Delta S = 1.8$~mm and $\Delta N = -0.38$~mm, and the approximate curves ignore the setbacks entirely. The slight difference between the exact and approximate shifts for $d = 10$~mm at the first fret can be eliminated if we include a term quadratic in $d$ in \eqn{q_n_approx}. As predicted above, we see that the dependence of $Q_n$ --- and therefore the tension shift --- on the setback values is minimal.

\begin{figure}
  \centering
  \includegraphics[width=5.0in]{figures/tnu_test}
  \caption{\label{fig:tnu_test} Comparison of the exact expression for the frequency shift caused by tension given by \eqn{tension_shift} with the approximate expression given by \eqn{tension_approx}.}
\end{figure}

Many luthiers provide ``relief'' to enlarge the effective height of the string (particularly for the wound bass strings) as the fret number grows to provide clearance for vibration amplitude at higher volume. In practice, this is accomplished by pivoting the fret board shown in \fig{guitar_schematic} clockwise about $x = X_0$, increasing the height of the string above fret $n$  by an amount
\begin{equation}
  \begin{split}
    \Delta y_n &= m \left(X_0 - X_n\right) \\
    &= \frac{\gamma_n - 1}{\gamma_n}\, m\, X_0 \\
    &= \frac{\gamma_n - 1}{\gamma_n}\, 2\, \Delta y_{12}\, ,
  \end{split}
\end{equation}
where $\Delta y_{12}$ is the relief at the twelfth fret and $m = 2\, \Delta y_{12} / X_0 \ge 0$ is the downward slope of the fret board. If we update \eqn{l_n_def} and \eqn{l_p_def} (with $d = 0$), then we obtain
\begin{subequations}
  \begin{align}
    L_n &= \sqrt{\left(X_n + \Delta S\right)^2 + (b + \Delta y_n + c)^2}\, , \nd \\
    L^\prime_n &= \sqrt{\left(X_0 - X_n + \Delta N\right)^2 + \left(b + \Delta y_n\right)^2}\, .
  \end{align}
\end{subequations}
These equations indicate that we could modify the approximation for $Q_n$ given by \eqn{q_n_approx} by replacing $b \longrightarrow b + \Delta y_n$, which results in the numerator
\begin{equation}
  \gamma_n\, b + (\gamma_n - 1)\, c \longrightarrow \gamma_n\, b + (\gamma_n - 1) \left(c + 2\, \Delta y_{12}\right)\, ,
\end{equation}
indicating that the intuitive substitution $c \longrightarrow c + 2\, \Delta y_{12}$ captures the effect of relief. (Note that this should \emph{not} be done when computing the length $L_0$ of the open string!)

% In \fig{tn_test}, we compare the exact expression for the frequency shift due to tension increases as a function of the fret number given by the \lhs of \eqn{tension_shift} with the approximate expression given by the \rhs. The exact curves for both $d = 0$~mm and $d = 10$~mm used the corresponding exact expressions for $\mathcal{L}_n$ and $L_0$, while the approximate expressions relied only on \eqn{q_n_approx}. Here we have chosen $\kappa = 51$ from \tbl{string_specs} and the guitar has the same parameters as in \fig{qn_test}.

\subsection{Bending Stiffness\label{sct:model_stiffness}}
% \begin{equation}
%f_{m n} = \frac{m}{2\, L_n}\, \sqrt{\frac{T_n}{\mu_n}} \left( 1 + B_n \right)\, ,
% \end{equation}
The bending stiffness of a string clamped at the $n^{\mathrm{th}}$ fret is given by \eqn{b_n_def}, \eqn{l_n_def}, and \eqn{t_n_def} as
\begin{equation} \label{eqn:bg_n_def}
  B_n = \sqrt{\frac{E\, \mathcal{A}\, s^2}{4\, T_n\, L_n^2}} = \sqrt{1 + \kappa\, Q_n}\, \frac{L_0}{L_n}\, \sqrt{\frac{E\, \mathcal{A}\, s^2}{4\, T_0\, L_0^2}} \approx \gamma_n\, B_0\, ,
\end{equation}
where the approximation applies when $B_0 \ll 1$ and the largest contribution arises from the shortened length of the fretted string compared to that of the open string. This expression confirms our intuitive expectation that the stiffness of the string should increase as the length becomes shorter.
% We see from \eqn{l_n_def} that $L_n \approx L_0/\gamma_n$, so from \eqn{b_n_def} we have
%  \begin{equation} %\label{eqn:bg_n_def}
% B_n = \sqrt{\frac{\pi\, \rho^4\, E}{4\, T_n\, L_n^2}} \approx \frac{L_0}{L_n}\, \sqrt{\frac{\pi\, \rho^4\, E}{4\, T_0\, L_0^2}} = \gamma_n\, B_0\, .
%  \end{equation}
Therefore, the fourth term in the final line of \eqn{error_def} can be approximated as
 \begin{equation} \label{eqn:dnu_bn}
1200\, \log_2 \left[ \frac{1 + B_n + (1 + \pi^2/2)\, B_n^2}{1 + B_0 + (1 + \pi^2/2)\, B_0^2} \right] \approx \frac{1200}{\ln(2)}\, \left[ \left(\gamma_n - 1\right) B_0 + \half\, \left(\gamma_n^2 - 1\right) \left(1 + \pi^2\right) B_0^2 \right]\, .
 \end{equation}
In \fig{bnu_test}, we use \eqn{dnu_bn} to compare the exact and approximate expressions for frequency shifts due to bending stiffness based on  \tbl{mcg_specs} and \tbl{string_specs}. Note that we show the approximate frequencies with and without the quadratic terms, and we see that the $2^\mathrm{nd}$-order contribution is about $0.2$~cents at the $12^\mathrm{th}$ fret. Once again, it is clear that $B_n$ does not depend significantly on either $\Delta S$ or $\Delta N$. In other words, the bending stiffness error does not depend on the tiny changes to the linear mass density or the tension that arise due to string fretting. Instead, it is an intrinsic mechanical property of the string: the stiffness increases as the length of the vibrating string becomes shorter. Comparing \fig{tnu_test} and \fig{bnu_test}, we see that at the 12$^\textrm{th}$ fret the frequency error due to bending stiffness is about twice as large as that caused by the increase in tension.

\begin{figure}
  \centering
  \includegraphics[width=5.0in]{figures/bnu_test}
  \caption{\label{fig:bnu_test} A comparison of exact and approximate expressions for the frequency shift due to bending stiffness, given by \eqn{dnu_bn}.}
\end{figure}

\subsection{Total Frequency Shift\label{sct:tot_freq_shift}}

Let's guide our intuition and prepare for the development of approximate expressions for $\Delta S$ and $\Delta N$ by relying on Taylor series expansions for all the effects described above. First, we'll ignore all quadratic terms in the resonant length error and adopt \eqn{rle_approx}. Next, we'll neglect the small reduction in linear mass density caused by fretting, and then rely on the approximation to the frequency shift caused by tension increases given by \eqn{tension_shift}. Finally, we'll describe the effects of bending stiffness using \eqn{dnu_bn}, \emph{neglecting} the term proportional to $B_0^2$. Incorporating all of these terms, we find that the total frequency shift is given approximately by
\begin{equation}\label{eqn:error_tot}
  \Delta \nu_n \approx \frac{1200}{\ln(2)}\, \left[ \left(\gamma_n - 1\right) \left(B_0 - \frac{\Delta S}{X_0}\right) + \frac{\Delta N}{X_0} + \half\, \kappa\, Q_n \right]\, .
\end{equation}
But how do we determine the bending stiffness $B_0$ given by \eqn{b_def} and the spring constant $\kappa$ given by \eqn{kappa_def} for a particular string?
% As shown in \fig{tfe_test}, this approximation is remarkably accurate for the Classical Guitar specified by \tbl{mcg_specs} (with nonzero setbacks) and a string with the parameters listed in \tbl{string_specs}.

% \begin{figure}
%   \centering
%   \includegraphics[width=5.0in]{figures/tfe_test}
%   \caption{\label{fig:tfe_test} A comparison of the total frequency shift given by \eqn{error_def} and \eqn{error_tot} for the Classical Guitar specified by \tbl{mcg_specs} (with nonzero setbacks) and a string with the parameters listed in \tbl{string_specs}.}
% \end{figure}

To measure $\kappa$, in \sct{exp} we will conduct an experiment that measures the change in the frequency of an open string as we make slight changes to its length~\cite{ref:byers1996cgi,ref:varieschi2010icf}. From \eqn{f_m_stiff}, the change $\delta f$ of the fundamental frequency of an open string due to a small  change in length $\delta L$ is
 \begin{equation}
 \begin{split}
\frac{\delta\, f}{\delta\, L} &= \frac{f}{L} \left( -1 + \frac{L}{2\, T}\, \frac{\delta\, T}{\delta\, L} - \frac{L}{2\, \mu}\, \frac{\delta\, \mu}{\delta\, L} + \frac{L}{1 + B}\, \frac{\delta\, B}{\delta\, L} \right) \\
&= \frac{f}{L} \left( -1 + \half\, \kappa + \half - \frac{B}{1 + B} \right) \\
&\approx \frac{f}{L} \times \half\, (\kappa - 1)\, ,
 \end{split}
 \end{equation}
where we have used the analyses above to determine that
 \begin{subequations}
 \begin{align}
\frac{\delta\, T}{\delta\, L} &= \frac{T}{L}\, \kappa\, , \\
\frac{\delta\, \mu}{\delta\, L} &= -\frac{\mu}{L}\, , \nd \\
\frac{\delta\, B}{\delta\, L} &= -\frac{B}{L}\, , \\
 \end{align}
 \end{subequations}
and we have again assumed that $B_0 \ll 1$. Therefore, following Byers~\cite{ref:byers1996cgi,ref:varieschi2010icf}, we define the parameter $R$ to be
 \begin{equation}\label{eqn:r_def}
R \equiv \frac{L}{f}\, \frac{\delta\, f}{\delta\, L} = \half\, (\kappa - 1)\, ,
 \end{equation}
which gives
 \begin{equation} \label{eqn:kappa_r}
\kappa = 2\, R + 1\, .
 \end{equation}

 We can anticipate the typical value of $R$ for a nylon classical guitar string through a simple observation. On a classical guitar with a scale length of 650~mm, we can usually tune an open string down a full step by winding the tuner/machine head five half turns down and then two half turns back up to re-tension the string. As we shall see in \sct{exp}, this decreases the effective string length by 3~mm. Since a full step is (by definition) 200~cents, \eqn{cents_approx} tells us that
\begin{equation}
  \frac{\Delta f}{f} \approx \frac{\ln(2)}{1200}\, \Delta \nu = \frac{200}{1731} = 0.116\, .
\end{equation}
In this case, we estimate $R$ to be
\begin{equation}
  R \approx \frac{650}{3}\, 0.116 = 25\, ,
\end{equation}
giving $\kappa \approx 51$, which are the values listed in \tbl{string_specs}. 

It's impractical to assume that the effective (differential) modulus of elasticity of a particular string can be derived from published values of bulk nylon (particularly in the case of a wound string). Instead, let's assume that we know the value of $\kappa$, and then estimate the bending stiffness coefficient by comparing \eqn{b_def} and \eqn{kappa_def}, and writing $B_0$ as
\begin{equation}
  B_0 = \sqrt{\kappa}\, \frac{s}{L_0}\, ,
\end{equation}
As discussed in \app{freq}, for a uniform cylindrical string/wire with radius $\rho$, $s = \rho/2$. This choice is valid for monofilament nylon strings~\cite{ref:woodland2004pgt}; if we provisionally accept it for wound nylon strings as well, then we have
\begin{equation} \label{eqn:b_0_kappa}
  B_0 = \sqrt{\kappa}\, \frac{\rho}{2\, L_0} \approx \sqrt{\kappa}\, \frac{\rho}{2\, X_0}\, .
\end{equation}
We'll test this phenomenological ansatz in \sct{exp}. We note that in the case of wound steel strings (often wrapped in metals like nickel or phosphor bronze), the bending stiffness depends on the radius of the nonuniform core alone~\cite{ref:fletcher1964nvf,ref:kemp2020ibg}. By contrast, the bass strings on a classical guitar are twisted and/or braided multifilament nylon strands wrapped in silver-plated copper.

In \fig{uncomp}, we compare the total frequency shifts predicted by \eqn{error_def} and \eqn{error_tot} for the Classical Guitar specified by \tbl{mcg_specs} with $\Delta S = \Delta N = 0$~mm, and a string with the parameters listed in \tbl{string_specs} at two different values of $d$. Note that the string is sharp at every fret, but even a large nonzero value of $d$ is only important at the first fret. The bending stiffness is negligible at the first fret, but accounts for 65\% of the shift at the 12$^\textrm{th}$ fret. The close agreement between the exact and approximate expressions for the frequency shifts gives us confidence that the equations we derive for the setbacks in \sct{comp} will be useful.

\begin{figure}
  \centering
  \includegraphics[width=5.0in]{figures/uncomp}
  \caption{\label{fig:uncomp} The total frequency shifts predicted by \eqn{error_def} and \eqn{error_tot} for the Classical Guitar specified by \tbl{mcg_specs} with $\Delta S = \Delta N = 0$~mm, and a string with the parameters listed in \tbl{string_specs}.}
\end{figure}

% \Eqn{error_tot} provides us with a clue on how to modify this guitar to improve the tone of this string. We see that the bending stiffness and the increase in string tension due to fretting sharpen the pitch, but that we can reduce these effects with a positive saddle setback and negative nut setback. In fact, a simple compensation strategy would be to choose
% \begin{subequations} \label{eqn:comp_approx}
%   \begin{align}
%     \Delta S &= B_0\, X_0 \approx \frac{\sqrt{\kappa}}{2}\, \rho\, , \nd \\
%     \Delta N &= - \kappa\, X_0\, \overline{Q} / 2\, ,
%   \end{align}
% \end{subequations}
% where we have applied \eqn{b_0_kappa} and defined $\overline{Q}$ as the relative displacement averaged over a particular set of frets to offset string stiffness and tension, respectively.  If we select the first twelve frets, then from \eqn{q_n_approx} with $d = 0$~mm we obtain
% \begin{equation} \label{eqn:qbar_approx}
%   \overline{Q} \approx \frac{77.9\, b^2 + 35.6\, b\, c + 5.82\, c^2}{24\, X_0^2}\, .
% \end{equation}
% In the case of the parameters used in \fig{uncomp}, we follow this simple approximate approach to estimate $\Delta S = 1.54$~mm, and $\Delta N = -0.51$~mm. We'll rely on a more accurate method for compensation in \sct{comp}, but we'll obtain results that are similar to those found using this basic approach. In particular, since a \emph{tunable} guitar string has $\sqrt{\kappa} \approx 7.1 \pm 0.3$, we'll discover that the saddle setback is primarily determined by the radius of the string.
