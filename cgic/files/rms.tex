%%%%%%%%%%%%%%%%%%%%%%%%%%%%%%%%%%%%%%%%%%%%%%%%%%%%%%%%%%%%%%%%%%%%%%%%%%%%%%
%
% Appendix file included in main project file using \input{}
%
% Assumes that LaTeX2e macros and packages defined in cg_comp.sty are
%   available
%
%%%%%%%%%%%%%%%%%%%%%%%%%%%%%%%%%%%%%%%%%%%%%%%%%%%%%%%%%%%%%%%%%%%%%%%%%%%%%%

 \section{Compensation by Minimizing RMS Error\label{app:rms}}

The root-mean-square (RMS) frequency error (in cents) averaged over the frets $n \in \{1, n_\text{max}\}$ (for $n_\text{max} > 1$) \emph{of a particular string} is given by
\begin{equation}\label{eqn:rms_def}
  \overline{\Delta \nu}_\text{rms} \equiv \sqrt{\frac{\sum_{n = 1}^{n_\text{max}} \Delta \nu_n^2}{n_\text{max}}}\, ,
\end{equation}
where $\Delta \nu_n$ is given by \eqn{error_def}. Here we will vary both $\Delta S$ and $\Delta N$ to minimize $\overline{\Delta \nu}_\text{rms}$. In this case, it is sufficient to minimize the quantity
 \begin{equation}\label{eqn:chi2_def}
\chi^2 = \sum_{n = 1}^{n_\text{max}} \left[\frac{\ln(2)}{1200}\, \Delta \nu_n\right]^2
 \end{equation}
such that the gradient of $\chi^2$ with respect to $\Delta S$ and $\Delta N$ vanishes. Let's rewrite \eqn{error_def} as
\begin{equation}\label{eqn:rms_dnuwz}
  \frac{\ln(2)}{1200}\, \Delta \nu_n = W_n + Z_n\, ,
\end{equation}
where
\begin{subequations}
  \begin{align}
    W_n &= \ln \left( \frac{L_0}{\gamma_n\, L_n} \right)\, , \nd \\
    Z_n &= \ln \left[ \sqrt{\frac{\mu_0}{\mu_n}\, \frac{T_n}{T_0}}\, \frac{1 + B_n + (1 + \pi^2/2)\, B_n^2}{1 + B_0 + (1 + \pi^2/2)\, B_0^2} \right]\, .
  \end{align}
\end{subequations}

In \sct{model}, we determined that --- for the purposes of estimating the values of the setbacks --- $W_n$ could be represented reasonably accurately by
\begin{equation} \label{eqn:w_n_approx}
  W_n \approx \frac{\Delta N - \left(\gamma_n - 1\right) \Delta S}{X_0}\, ,
\end{equation}
but for completeness we'll add the term in \eqn{rle_taylor} that is quadratic in $b$ and $c$ to $Z_n$. Furthermore, we discovered that $Z_n$ does not depend to second order on either $\Delta S$ or $\Delta N$. Therefore, the components of the gradient of $\chi^2$ are
\begin{subequations} \label{chi2_grad}
  \begin{align}
    \frac{\partial}{\partial \Delta S}\, \chi^2 &= 2 \sum_n \left(W_n + Z_n\right)\frac{\partial W_n}{\partial \Delta S} = -\frac{2}{X_0}\, \sum_n (\gamma_n - 1) \left(W_n + Z_n\right)\, , \nd \\
    \frac{\partial}{\partial \Delta N}\, \chi^2 &= 2 \sum_n \left(W_n + Z_n\right)\frac{\partial W_n}{\partial \Delta N} = \frac{2}{X_0}\, \sum_n \left(W_n + Z_n\right)\, .
  \end{align}
\end{subequations}
Setting each of these expressions to zero and solving them for $\Delta S$ and $\Delta N$, we obtain
\begin{subequations} \label{eqn:rms_sol}
  \begin{align}
    \label{eqn:rms_sol_ds} \Delta S &= \frac{g_0\, \overline{\mathcal{Z}}_1 - g_1\, \overline{\mathcal{Z}}_0}{g_0\, g_2 - g_1^2}\, X_0\, , \nd \\
    \label{eqn:rms_sol_dn} \Delta N &= -\frac{g_2\, \overline{\mathcal{Z}}_0 - g_1\, \overline{\mathcal{Z}}_1}{g_0\, g_2 - g_1^2}\, X_0\, ,
  \end{align}
\end{subequations}
where
\begin{align}
  g_k &\equiv \sum_{n = 1}^{n_\text{max}} \left(\gamma_n - 1\right)^k\, , \nd \\
  \overline{\mathcal{Z}}_k &\equiv \sum_{n = 1}^{n_\text{max}} \left(\gamma_n - 1\right)^k\, Z_n\, .
\end{align}

The corresponding Hessian matrix for this problem is the symmetric matrix
 \begin{equation}
H = \begin{bmatrix}
      \frac{\partial^2 \chi^2}{\partial \Delta S^2} & \frac{\partial^2 \chi^2}{\partial \Delta N\, \partial \Delta S} \\
      \frac{\partial^2 \chi^2}{\partial \Delta S\, \partial \Delta N} & \frac{\partial^2 \chi^2}{\partial \Delta N^2}
    \end{bmatrix}
  = \frac{2}{X_0^2} \begin{bmatrix}
      g_2 & -g_1 \\
      -g_1 & g_0
    \end{bmatrix}\, .
 \end{equation}
We can apply the second partial derivative test to the Hessian to determine whether we've found an extremum of $\chi^2$. If the determinant of the Hessian is positive, and (in the case of a $2 \times 2$ matrix) one of the diagonal elements is positive, then we have found a minimum. The determinant is greater than zero for $n_\text{max} \ge 2$, and the second condition is satisfied by $g_0 = n_\text{max} > 0$ when $n_\text{max} \ge 1$. Therefore, we can be confident that the solution for $\Delta S$ and $\Delta N$ given by \eqn{rms_sol} minimizes $\chi^2$ accurately to first order in $\Delta S$ and $\Delta N$ provided that we are averaging over at least the first two frets. Note that the diagonal elements of the Hessian also allow us to estimate the increase in the residual RMS frequency error caused by small changes $\delta s$ and $\delta n$ in the saddle and nut setbacks respectively; we obtain
\begin{equation}
  \overline{\delta \nu}_\text{rms} = \frac{1}{n_\text{max}\, \overline{\Delta \nu}_\text{rms}} \left[ \frac{1200}{\ln(2)} \right]^2 \left[ g_2 \left(\frac{\delta s}{X_0}\right)^2 + g_0 \left(\frac{\delta n}{X_0}\right)^2\right]\, .
\end{equation}

We can further refine the predicted values of these setbacks to accommodate the small second-order terms in $\Delta S$ and $\Delta N$ neglected in the resonant length error approximation used in \eqn{w_n_approx}. Relying on \eqn{error_def} as the exact expression for the frequency error $\Delta \nu_n$, we can use \eqn{rms_sol} to provide initial values for a nonlinear minimization of $\sum_n\, \Delta \nu_n^2$ over the first 12 frets. We adopt the quasi-Newton algorithm of Broyden, Fletcher, Goldfarb, and Shanno~\cite{ref:nocedal2006no}, a second-order algorithm for numerical optimization. Typically, this additional step changes the setback values by only a fraction of a percent. We'll refer to this approach as the ``RMS Minimize'' method, and we use it throughout this work to compute the setbacks for each string under study. Note that the approximate equations given by \eqn{comp_approx} also can be used to compute initial values for this final nonlinear minimization.

% Again, to guide our intuition, we follow the logic used to develop \eqn{error_tot} and approximate $Z_n$ as
% \begin{equation}
%   Z_n \approx \half\, \kappa\, Q_n + \left(\gamma_n - 1\right) B_0\, .
% \end{equation}
% Then our estimates for the setbacks become
% \begin{subequations}
%   \begin{align}
%     \Delta S &= \left(B_0 + \frac{\kappa}{2}\, \frac{g_0\, \overline{\mathcal{Q}}_1 - g_1\, \overline{\mathcal{Q}}_0}{g_0\, g_2 - g_1^2} \right) X_0\, , \nd \\
%     \Delta N &= -\frac{\kappa}{2}\, \frac{g_2\, \overline{\mathcal{Q}}_0 - g_1\, \overline{\mathcal{Q}}_1}{g_0\, g_2 - g_1^2}\, X_0\, ,
%   \end{align}
% \end{subequations}
% where
% \begin{equation}
%   \overline{\mathcal{Q}}_k \equiv \sum_{n = 1}^{n_\text{max}} \left(\gamma_n - 1\right)^k\, Q_n\, .
% \end{equation}
% If we approximate $\overline{\mathcal{Q}}_k \approx g_k\, \overline{Q}$, where $\overline{Q}$ is the relative displacement averaged over a particular set of frets, we obtain the estimates discussed in \sct{tot_freq_shift}: $\Delta S \approx B_0\, X_0$, and $\Delta N \approx -\kappa\, \overline{Q}\, X_0 / 2$. The corresponding solution when the quadratic stiffness term is included is given in matrix form by
% \begin{equation}\label{eqn:rms_sol_quad}
% \begin{bmatrix}
%   \Delta S \\
%   \Delta N
% \end{bmatrix} = \frac{X_0}{g_0\, g_2 - g_1^2}\,
% \begin{bmatrix}
%   g_0 & -g_1 \\
%   g_1 & -g_2
% \end{bmatrix}\,
% \begin{bmatrix}
%   g_2\, B_0 + \half \left(1 + \pi^2\right) \left(2 g_2 + g_3\right) B_0^2 + \half\, \kappa\, \overline{Q}_1 \\
%   g_1\, B_0 + \half \left(1 + \pi^2\right) \left(2 g_1 + g_2\right) B_0^2 + \half\, \kappa\, \overline{Q}_0
% \end{bmatrix}\, .
%  \end{equation}
% If we choose $n_\text{max} = 12$, then when $d = 0$ we can compute $g_k$ and $\overline{Q}_k$ in terms of powers of $b$ and $c$ to find
% \begin{align} \label{eqn:rms_sol_comp}
%   \Delta S &= \left[ B_0 + \frac{3}{2} \left(1 + \pi^2\right) B_0^2 \right] X_0 + \frac{\kappa}{4\, X_0} \left( -9.848\, b^2 + 2\, b\, c + c^2 \right)\, , \nd \\
%   \Delta N &= B_0^2\, X_0 - \frac{\kappa}{2\, X_0} \left( 5.633\, b^2 + b\, c \right)\, .
% \end{align}


%  \begin{subequations}
%  \begin{align}
% \frac{\partial}{\partial \Delta S}\, \chi^2 &= -\frac{2}{X_0}\, \sum_n (\gamma_n - 1)\left[ (\gamma_n - 1) \left(B_0 - \frac{\Delta S}{X_0}\right) + \frac{\Delta N}{X_0} + \frac{\kappa}{2}\, Q_n \right]\, , \nd \\
% \frac{\partial}{\partial \Delta N}\, \chi^2 &= \frac{2}{X_0}\, \sum_n \left[ (\gamma_n - 1) \left(B_0 - \frac{\Delta S}{X_0}\right) + \frac{\Delta N}{X_0} + \frac{\kappa}{2}\, Q_n \right]\, .
%  \end{align}
%  \end{subequations}
% Setting both of these expressions to zero, we can rewrite them as the matrix equation
%  \begin{equation}% \label{eqn:rms_mat}
% \begin{bmatrix}
%   g_2 & -g_1 \\
%   g_1 & -g_0
% \end{bmatrix}\,
% \begin{bmatrix}
%   \Delta S \\
%   \Delta N
% \end{bmatrix} = X_0
% \begin{bmatrix}
%   g_2\, B_0 +  \half\, \kappa\, \overline{Q}_1 \\
%   g_1\, B_0 +  \half\, \kappa\, \overline{Q}_0
% \end{bmatrix}\, ,
%  \end{equation}
% where
%  \begin{align}
% g_k &\equiv \sum_{n = 1}^{n_\text{max}} \left(\gamma_n - 1\right)^k\, , \nd \\
% \overline{Q}_k &\equiv \sum_{n = 1}^{n_\text{max}} \left(\gamma_n - 1\right)^k\, Q_n\, .
%  \end{align}
% We note that
% \begin{equation}
%   g_k \equiv \sum_{n = 1}^{n_\text{max}} \gamma_n^k = \frac{\gamma_k \left(\gamma_{k n_\text{max}} - 1\right)}{\gamma_k - 1}\, ,
% \end{equation}
% and therefore
% \begin{subequations}
%   \begin{align}
%     g_0 &= n_\text{max}\, ,\\
%     g_1 &= g_1 - n_\text{max}\, , \nd \\
%     g_2 &= g_2 - 2 g_1 + n_\text{max}\, .
%   \end{align}
% \end{subequations}

%  \Eqn{rms_mat} has the straightforward analytic solution
% \begin{equation}
%   \begin{bmatrix}
%     \Delta S \\
%     \Delta N
%   \end{bmatrix} = \frac{X_0}{g_0\, g_2 - g_1^2}\,
%   \begin{bmatrix}
%     g_0 & -g_1 \\
%     g_1 & -g_2
%   \end{bmatrix}\,
%   \begin{bmatrix}
%     g_2\, B_0 + \half\, \kappa\, \overline{Q}_1 \\
%     g_1\, B_0 + \half\, \kappa\, \overline{Q}_0
%   \end{bmatrix}\, ,
% \end{equation}
% or
% \begin{subequations}
%   \begin{align}
%     \Delta S &= \left(B_0 + \frac{\kappa}{2}\, \frac{g_0\, \overline{Q}_1 - g_1\, \overline{Q}_0}{g_0\, g_2 - g_1^2} \right) X_0\, , \nd \\
%     \Delta N &= -\frac{\kappa}{2}\, \frac{g_2\, \overline{Q}_0 - g_1\, \overline{Q}_1}{g_0\, g_2 - g_1^2}\, X_0\, .
%   \end{align}
% \end{subequations}
% If we approximate $\overline{Q}_k \approx g_k\, \overline{Q}$, where $\overline{Q}$ is the relative displacement averaged over a particular set of frets, we obtain the estimates discussed in \sct{tot_freq_shift}.

The setback solution given by \eqn{rms_sol} is valid for a single string, and results like those shown in \tbl{ej45_setbacks} and \fig{shift_classicalguitar_ej45_full} assume that the guitar is built such that each string --- from a particular set of strings --- has a unique saddle and nut setback. Suppose that we'd prefer to engineer a guitar with single, uniform values of both $\Delta S$ and $\Delta N$ that provide reasonable compensation across an entire string set (or an ensemble of strings from a variety of manufacturers). In this case, \eqn{rms_def} becomes
 \begin{equation}\label{eqn:rms_def_m}
\overline{\Delta \nu}_\text{rms} \equiv \sqrt{\frac{\sum_{m = 1}^{m_\text{max}} \sum_{n = 1}^{n_\text{max}} \left[\Delta \nu^{(m)}_{n}\right]^2}{m_\text{max}\, n_\text{max}}}\, ,
 \end{equation}
where $m$ labels the strings in the set, and \eqn{rms_dnuwz} has been updated to become
%  \begin{equation}\label{eqn:error_tot_m}
% \Delta \nu^{(m)}_n \approx \frac{1200}{\ln(2)}\, \left\{ \left(\gamma_n - 1\right) \left[B_0^{(m)} - \frac{\Delta S}{X_0}\right] + \frac{\Delta N}{X_0} + \half\, \kappa^{(m)}\, Q_n \right\}\, .
%  \end{equation}
\begin{equation}%\label{rms_dnuwz}
  \frac{\ln(2)}{1200}\, \Delta \nu^{(m)}_n = W^{(m)}_n + Z^{(m)}_n\, ,
\end{equation}
If we rigorously follow the same approach that we used to arrive at \eqn{rms_sol}, in the multi-string case we obtain
\begin{subequations}
  \begin{align}
    \Delta S &= \frac{1}{m_\text{max}}\, \sum_{m = 1}^{m_\text{max}} \Delta S^{(m)}\, , \nd \\
    \Delta N &= \frac{1}{m_\text{max}}\, \sum_{m = 1}^{m_\text{max}} \Delta N^{(m)}\, ,
  \end{align}
\end{subequations}
% \begin{equation}\label{eqn:rms_sol_multi}
% \begin{bmatrix}
%   \Delta S \\
%   \Delta N
% \end{bmatrix} = \frac{1}{m_\text{max}}\, 
% \begin{bmatrix}
%   \sum_{m = 1}^{m_\text{max}} \Delta S^{(m)} \\
%   \sum_{m = 1}^{m_\text{max}} \Delta N^{(m)}
% \end{bmatrix}\, ,
%  \end{equation}
where
\begin{subequations}% \label{eqn:rms_sol}
  \begin{align}
    \Delta S^{(m)} &= \frac{g_0\, \overline{\mathcal{Z}}^{(m)}_1 - g_1\, \overline{\mathcal{Z}}^{(m)}_0}{g_0\, g_2 - g_1^2}\, X_0\, , \nd \\
    \Delta N^{(m)} &= -\frac{g_2\, \overline{\mathcal{Z}}^{(m)}_0 - g_1\, \overline{\mathcal{Z}}^{(m)}_1}{g_0\, g_2 - g_1^2}\, X_0\, ,
  \end{align}
\end{subequations}
% \begin{equation}\label{eqn:rms_sol_uni}
% \begin{bmatrix}
%   \Delta S^{(m)} \\
%   \Delta N^{(m)}
% \end{bmatrix} = \frac{X_0}{g_1^2 - g_0\, g_2}\,
% \begin{bmatrix}
%   -g_0 & g_1 \\
%   -g_1 & g_2
% \end{bmatrix}\,
% \begin{bmatrix}
%   g_2\, B_0^{(m)} + \half\, \kappa^{(m)}\, \overline{Q}_1 \\
%   g_1\, B_0^{(m)} + \half\, \kappa^{(m)}\, \overline{Q}_0
% \end{bmatrix}\, .
%  \end{equation}
reflecting the unique values of $\kappa^{(m)}$ and $B_0^{(m)}$ for each string in each set. In other words, we can find the optimum values for both $\Delta S$ and $\Delta N$ simply by averaging the corresponding setbacks over a commercially interesting collection of string sets. 