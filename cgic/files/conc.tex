%%%%%%%%%%%%%%%%%%%%%%%%%%%%%%%%%%%%%%%%%%%%%%%%%%%%%%%%%%%%%%%%%%%%%%%%%%%%%%
%
% Section file included in main project file using \input{}
%
% Assumes that LaTeX2e macros and packages defined in cg_comp.sty are
%   available
%
%%%%%%%%%%%%%%%%%%%%%%%%%%%%%%%%%%%%%%%%%%%%%%%%%%%%%%%%%%%%%%%%%%%%%%%%%%%%%%

 \section{Conclusion: The Recipes\label{sct:conc}}

 In this work, we have constructed a model of classical guitar intonation that includes the effects of the resonant length of the fretted string, linear mass density, tension, and bending stiffness. We have described a simple experimental approach to estimating the increase in string tension arising from an increase in its length, and then the corresponding mechanical stiffness. This allows us to determine the saddle and nut positions needed to compensate the guitar for a particular string, and we propose a simple approach to find averages of these positions to accommodate a variety of strings. This ``mean'' method benefits further from temperament techniques --- such as harmonic tuning --- that can enhance the intonation of the classical guitar for particular musical pieces.

Our calculations have relied on \eqn{f_m_stiff}, which was derived by compromising for empirical reasons on symmetric boundary conditions and assuming that the string was pinned to the saddle rather than clamped. We then separated the contributions to the frequency deviations from ideal values caused by fretting by expressing these differences using the definition of logarithmic ``cents'' given by \eqn{cents_def}, resulting in the analytically exact expression for nonideal frequency shifts given by \eqn{error_def}. We have used this equation to plot frequency errors at each of the first twelve frets for a prototypical Classical Guitar with a variety of compensation strategies based on an  RMS fit method described in \app{rms}. Because the height of each string above the frets is small compared to the scale length, there are Taylor series approximations of the terms in \eqn{error_def} that we used to derive \eqn{error_tot} to guide our understanding of the underlying principles of guitar compensation. This intuition led us to approximate estimates of the ideal values of the saddle and nut setbacks given quite accurately by \eqn{comp_approx}. These setback estimates can be averaged across the string set to design compensated nuts and saddles that should be relatively easy to fabricate. Nevertheless, we understand that high-end (concert) guitars that are likely to rely on one or two string sets (and the appropriate value of $d$ for one guitar player) will benefit from the full, more accurate treatment of individual string setbacks.

From these results, we have been able to create two ``recipes'' --- based on the RMS fit method and the Taylor-series-based approximation --- that predict saddle and nut setbacks which enable the guitar to compensate for the frequency effects of fretting. Applying either of these algorithms to a particular guitar design always begins with the same five steps:
\begin{enumerate}
    \item Determine the scale length of the guitar by doubling the distance between the inside edge of the nut and the center of the $12^\textrm{th}$ fret. 
    \item Using \fig{guitar_schematic} as a guide, carefully measure the values of $b$ and $c$. It is possible that the luthier has selected a saddle with vertical curvature, resulting in different values of $c$ for each string.
    \item Estimate the relief $\Delta y_{12}$ at the $12^\textrm{th}$ fret for each string. Measure the action (height) $y_{12}$ of the string above fret 12; then $\Delta y_{12} = y_{12} - b - c/2$ and we rescale the height of the saddle above the nut to $c + 2\, \Delta y_{12}$.
    \item Select a string set with values of $\kappa$ and $B_0$ listed in one of the derived physical properties tables in this paper, or follow the procedure developed in \sct{exp} to determine these quantities for a different string set.
    \item Referring to \fig{guitar_schematic} as a guide, choose a preferred value of the fretting distance $d$ to account for the size of the finger.
\end{enumerate}
The most accurate algorithm then adds one more step:
\begin{enumerate}
    \setcounter{enumi}{5}
    \item Use \eqn{rms_sol} to determine the saddle and nut setbacks for the selected string set.
\end{enumerate}
Alternatively, a promising approximate approach can be followed:
\begin{enumerate}
    \setcounter{enumi}{5}
    \item Use \eqn{comp_approx} to compute the saddle and nut setbacks for the selected string set.
\end{enumerate}
The BFGS nonlinear minimization method discussed in \app{rms} can be used to refine either of these estimates further. But perhaps the simplest reasonably accurate compensation approach is to adopt the results of \fig{dsdn_mean}: compute the saddle setback using $\Delta S = 4.4\, \overline{\rho}$, and select the nut setback to be $\Delta N = -0.38 \times (650 \textrm{ mm} / X_0)$~mm, or about $15$~mils for a guitar with a scale length of $650$~mm.

In the future, it could be worthwhile to study further the boundary conditions that result in the coefficient of the linear and quadratic $B$ terms in \eqn{f_m_stiff}, taking into consideration the polarization of the string's vibration. We measured the frequency deviations of monofilament strings at the twelfth fret of several guitars and were able to rule out a factor of 2, but a more precise value would result in more accurate predictions of the saddle setback. (We speculate that this coefficient may depend on the construction of the saddle.) Similarly, we measured the correct value for the radius of gyration of wound nylon strings to be $\rho/2$ with a 30\% standard deviation, which leaves some room for improvement. A numerical simulation using multiphysics software may be able to refine this value further. Finally, we are at a loss to understand the high $R$ values of the light-tension string set that we measured; since they are not significantly different in volume mass density than the other sets we studied, we suspect that there is a different manufacturing process (such as chemical composition) at play.

We have placed the text of this manuscript (as well as the computational tools needed to reproduce our numerical results and all the graphs presented here) online~\cite{ref:github2024rgb} to invite comment and contributions from and collaboration with interested luthiers and musicians.