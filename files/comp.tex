%%%%%%%%%%%%%%%%%%%%%%%%%%%%%%%%%%%%%%%%%%%%%%%%%%%%%%%%%%%%%%%%%%%%%%%%%%%%%%
%
% Section file included in main project file using \input{}
%
% Assumes that LaTeX2e macros and packages defined in cg_comp.sty are
%   available
%
%%%%%%%%%%%%%%%%%%%%%%%%%%%%%%%%%%%%%%%%%%%%%%%%%%%%%%%%%%%%%%%%%%%%%%%%%%%%%%

 \section{Classical Guitar Compensation\label{sct:comp}}

As we discussed in \sct{tot_freq_shift}, \eqn{error_tot} provides a guide to how to modify our prototypical classical guitar to improve the tone of the string shown in \fig{uncomp}. We noted that the bending stiffness and the increase in string tension due to fretting sharpen the pitch, but that we can flatten it with a positive saddle setback and negative nut setback. As in \fig{uncomp}, let's choose $b = 1$~mm, $c = 4$~mm, $R = 24$ and $\rho = 0.5$~mm as in \fig{uncomp}. Then the corresponding string constant and bending stiffness are $\kappa = 49$ and $B_0 = 0.00268$ respectively. In this case, \Fig{comp_dsdn} shows that increasing the saddle setback tends to rotate the pitch curve downward, and increasing the magnitude of the negative nut setback displaces the pitch curve almost uniformly downward. In this context --- referring to \eqn{error_tot} --- we follow \sct{tot_freq_shift} and choose $\Delta S = B_0\, X_0$ to compensate for stiffness, and then select $\Delta N = - \kappa\, X_0\, \overline{Q} / 2$, where $\overline{Q}$ is the relative displacement averaged over a particular set of frets. If we compute this mean over the first $12$ frets, then we estimate $\Delta S = 1.75$~mm, $\Delta N = -0.49$~mm for $d = 0$, and $\Delta N = -0.55$~mm for $d = 10$. The pitch curve shown in \fig{comp_est} illustrates the dramatic improvement in tone that can be obtained via classical guitar compensation.

\begin{figure}
    \centering
    \begin{subfigure}[b]{0.8\textwidth}
        \centering
        \includegraphics[width=5.0in]{../figures/comp_ds}
        \caption{Frequency shifts ($\Delta N = 0$)}
        \label{fig:comp_ds}
    \end{subfigure}
    \par\vspace{0.25in}
    \begin{subfigure}[b]{0.8\textwidth}
        \centering
        \includegraphics[width=5.0in]{../figures/comp_dn}
        \caption{Frequency shifts ($\Delta S = 0$)}
        \label{fig:comp_dn}
    \end{subfigure}
    \caption{\label{fig:comp_dsdn} In (a), we plot the frequency shifts for our classical guitar for several saddle setbacks with $\Delta N = 0$. Here $R = 24$ and the string radius is 0.5~mm. In (b), we set $\Delta S = 0$ and plot the frequency shifts for several nut ``setbacks.''}
  \end{figure}
  
  \begin{figure}
    \centering
    \includegraphics[width=5.0in]{../figures/comp_est}
    \caption{\label{fig:comp_est} The total frequency shift given by \eqn{error_def} for a classical guitar with a scale length of 650~mm, $b = 1.0$~mm, $c = 4.0$~mm, and a string with $R = 24$ and a radius $\rho = 0.5$~mm.}
  \end{figure}

% Using the data and results presented in \tbl{ej45_props}, we can explore different approaches to compensating guitar strings for bending stiffness and string tension perturbations. For example, consider the guitar string shown in 

% For example, in \fig{shift_classicalguitar_ej45_factory}, using \eqn{error_def} we plot the frequency deviation (in cents) from ideal 12-TET for each string at each of the first 12 frets of an Alhambra 8P classical guitar, assuming that the open string has been perfectly tuned to the correct frequency. Recall that the Alhambra 8P is manufactured with a saddle setback of 1.5~mm, presumably to offset the effects of bending stiffness in the strings. For comparison, in \fig{shift_classicalguitar_ej45_null}, we plot the same deviations for the case where $\Delta S = 0$, which increases the error of each string at the 12$^\text{th}$ fret by 3 -- 4 cents. Recall from \sct{model} that we could crudely predict the values of the saddle and nut setbacks by inspecting \eqn{error_tot}. For example, from \tbl{ej45_props}, we estimate $\Delta S \approx B_0\, X_0 = 2.7$~mm and $\Delta N \approx -1.5$~mm for the third (G) string.

\begin{figure}
  \centering
  \includegraphics[width=5.0in]{../figures/shift_classicalguitar_ej45_null}
  \caption{\label{fig:shift_classicalguitar_ej45_null} Frequency errors for an uncompensated classical guitar with a scale length of 650~mm, $b = 1.0$~mm, $c = 4.0$~mm, $d = 0$~mm, and normal tension nylon strings (D'Addario EJ45).}
\end{figure}

\begin{table}%[htbp]
  \centering
  \caption{\label{tbl:ej45_setbacks} Predicted setbacks for the D'Addario Pro-Arte Nylon Classical Guitar Strings -- Normal Tension (EJ45) on the Alhambra 8P classical guitar.}
  \begin{tabular}{cccc}
\toprule
String &  $\Delta S$ (mm) &  $\Delta N$ (mm) &  $\overline{\Delta \nu}_\text{rms}$ (cents) \\
\midrule
 J4501 &             2.16 &            -0.43 &                                        0.18 \\
 J4502 &             1.92 &            -0.31 &                                        0.15 \\
 J4503 &             4.36 &            -0.82 &                                        0.31 \\
 J4504 &             1.30 &            -0.19 &                                        0.13 \\
 J4505 &             1.94 &            -0.28 &                                        0.15 \\
 J4506 &             2.62 &            -0.35 &                                        0.16 \\
\bottomrule
\end{tabular}


\end{table}%

As an alternative to this simple approach, we adopt the method described in \app{rms} and adjust the setbacks to minimize the root-mean-squared (RMS) average of the frequency deviations for each string. This mean (over the first 12 frets) can be computed by squaring the frequency deviations shown in \fig{shift_classicalguitar_ej45_null}, averaging those values, and then taking the square root of the result. Using \eqn{rms_sol_quad}, the setbacks we obtain are listed in \tbl{ej45_setbacks}, and the corresponding frequency deviations --- obtained with the listed setback for each string --- are shown in \fig{shift_classicalguitar_ej45_full} (assuming that all other aspects of the guitar remain unchanged). Of course, manufacturing a guitar with unique saddle and nut setbacks for each string (of a particular tension) can be challenging, so we also plot in \fig{shift_classicalguitar_ej45_mean} the shifts obtained by setting single values of $\Delta S$ and $\Delta N$ to the mean of the corresponding column in  \tbl{ej45_setbacks}. Note that the saddle setbacks tend to be larger --- and the nut setbacks smaller --- than the simple estimates that we made above. This is easily understood by examining \eqn{error_tot}: the portion of $\Delta S$ that exceeds $B_0\, X_0$ scales with $\gamma_n - 1$, and helps to compensate for tension errors as $n$ increases.

%  \begin{figure}
%   \centering
%   \begin{subfigure}[b]{0.8\textwidth}
%    \centering
%    \includegraphics[width=5.0in]{../figures/shift_classicalguitar_ej45_factory}
%    \caption{Factory guitar}
%    \label{fig:shift_classicalguitar_ej45_factory}
%   \end{subfigure}
%   \par\vspace{0.25in}
%   \begin{subfigure}[b]{0.8\textwidth}
%    \centering
%    \includegraphics[width=5.0in]{../figures/shift_classicalguitar_ej45_null}
%    \caption{Uncompensated}
%    \label{fig:shift_classicalguitar_ej45_null}
%   \end{subfigure}
%   \caption{\label{fig:classicalguitar_ej45} Frequency shifts (in cents) for an Alhambra 8P guitar with normal tension nylon strings (D'Addario EJ45). In (a), we show the deviations of the guitar as manufactured in the factory, completely consistent with our measurements. In (b), we show the same 12-TET errors that would arise if $\Delta S = 0$ for the same guitar.}
%  \end{figure}

 \begin{figure}
  \centering
  \begin{subfigure}[b]{0.8\textwidth}
   \centering
   \includegraphics[width=5.0in]{../figures/shift_classicalguitar_ej45_full}
   \caption{Full compensation}
   \label{fig:shift_classicalguitar_ej45_full}
  \end{subfigure}
  \par\vspace{0.25in}
  \begin{subfigure}[b]{0.8\textwidth}
   \centering
   \includegraphics[width=5.0in]{../figures/shift_classicalguitar_ej45_mean}
   \caption{Mean compensation}
   \label{fig:shift_classicalguitar_ej45_mean}
  \end{subfigure}
  \caption{\label{fig:compensation_classicalguitar_ej45} Frequency shifts (in cents) for a classical guitar with normal tension nylon strings (D'Addario EJ45). In (a) we use the individual values for each string that are listed in \tbl{ej45_setbacks}. In (b), we set $\Delta S$ and $\Delta N$ to the mean of the corresponding column in that table.}
 \end{figure}

We close this section with several comments. First, as mentioned above, it is nontrivial to manufacture a guitar with different setbacks for each string~\cite{ref:byers1996cgi}, and it is unlikely that the exact values listed in \tbl{ej45_setbacks} are applicable to other string sets. We have measured the values of $R$ for five other string sets (with $d = 0$), and in \app{specs} we have reproduced the compensation procedure for them. Although each set exhibits variation between strings (and with respect to other sets) in individual setbacks for each string, they are similar enough that we suspect that there is the potential for great simplification in guitar design. For example, following the analysis of \app{rms}, it is possible to determine a single setback pair $\{\Delta S, \Delta N\}$ that minimizes the RMS frequency errors of an ensemble of strings over a collection of frets simply by computing the mean of the setbacks over all strings, and then using these mean values when manufacturing the guitar. If we consider five of the six string sets we have measured here --- neglecting the light tension strings because of their pathologically high values of $R$ --- we obtain the global mean setbacks $\Delta S = 1.90$~mm and $\Delta N = -0.38$~mm for the saddle and nut, respectively. Note that these results are remarkably similar to the values used in \fig{shift_classicalguitar_ej45_mean}; if we plot the frequency deviations of those five string sets with these particular setback values, then we find that the maximum error always occurs at the twelfth fret, and it is always less than 2~cents. In the next section, we discuss a method to temper the guitar to reduce these errors further.

Second, after paying close attention to the impact of the act of fretting on our approximations in \sct{model}, we have simply set $d = 0$ in this section. This is because nonzero values of $d$ have an impact on the frequency deviation of a string at the first fret, and is otherwise negligible. This tends to increase the required mean value of the nut setback, but does not require a significant change to the saddle setback. As an example, in \fig{dsdnd_ej45}, we plot the mean optimum saddle and nut setbacks for the classical guitar parameters used in \fig{shift_classicalguitar_ej45_mean} with the D'Addario Nylon Normal Tension EJ45 string set as a function of the fretting distance $d$. Paying careful attention to the $y$-axis of these plots, we see that the value of the mean saddle setback \emph{decreases} by less than 5\% as $d$ increases to 10~mm, and the magnitude of the nut setback increases by almost 30\% at the same fretting offset. This behavior is essentially the same for all string sets considered in this work, and can be used by the luthier to determine the optimum setback values for their designs. Note that --- when the RMS approach is used to compute the mean setbacks --- the difference $\Delta S - \Delta N$ (the sum of the saddle setback and the magnitude of the nut setback) is virtually independent of $d$, as shown in \fig{dsnd_ej45}.

\begin{figure}
  \centering
  \begin{subfigure}[b]{0.8\textwidth}
      \centering
      \includegraphics[width=5.0in]{../figures/dsd_ej45}
      \caption{Mean saddle setback}
      \label{fig:dsd_ej45}
  \end{subfigure}
  \par\vspace{0.25in}
  \begin{subfigure}[b]{0.8\textwidth}
      \centering
      \includegraphics[width=5.0in]{../figures/dnd_ej45}
      \caption{Mean nut setback}
      \label{fig:dnd_ej45}
  \end{subfigure}
  \caption{\label{fig:dsdnd_ej45} The mean optimum saddle and nut setbacks for the D'Addario Nylon Normal Tension EJ45 string set as a function of the fretting distance $d$.}
\end{figure}

\begin{figure}
  \centering
  \includegraphics[width=5.0in]{../figures/dsnd_ej45}
  \caption{\label{fig:dsnd_ej45} The mean value of $\Delta S - \Delta N$ for the D'Addario Nylon Normal Tension EJ45 string set as a function of the fretting distance $d$.}
\end{figure}

\begin{figure}
  \centering
  \includegraphics[width=5.0in]{../figures/japan_guitar_ej45_shifts}
  \caption{\label{fig:japan_guitar_ej45_shifts} Expected frequency shifts for a classical guitar using normal-tension nylon strings with $b = 1.4$~mm, $c = 5.5$~mm, $d = 10$~mm, $X_0 = 652$~mm, $\Delta S = -3.5$~mm, and $\Delta N = 0$~mm.}
\end{figure}

Third, recall that we recalculated the expected frequency shift of a classical guitar string with asymmetric boundary conditions in \app{freq}, and found an expression for $f_q$ given by \eqn{f_m_hybrid} that reduces --- by about a factor of 2 --- the impact of the bending stiffness relative to the symmetric boundary condition case given by the symmetric (clamped) case in \eqn{f_m_clamped}. Furthermore, we have not used more sophisticated techniques to calculate the bending stiffness for either monofilament nylon strings or wound nylon strings, opting instead for the phenomenological model given by \eqn{b_0_kappa}. But is this approach valid? As a test, we compared our frequency shift estimates based on \sct{model} with experimental measurements made using a guitar manufactured in Japan based on a few unique design choices: $b = 1.4$~mm, $c = 5.5$~mm, $d = 10$~mm, $X_0 = 652$~mm, $\Delta S = -3.5$~mm, and $\Delta N = 0$~mm. The larger value of $b$ and the large \emph{negative} saddle setback result in very large frequency deviations at all frets. We measured the shifts at the first fret and found that they fell into the range of $4.5 - 5.5$~cents, consistent with our predictions in \fig{japan_guitar_ej45_shifts}. At the 12$^\mathrm{th}$ fret, we measured $\Delta f = 18$~cents for the third and sixth strings, and $\Delta f = 14.5 - 16$~cents for the other strings, in reasonably close agreement with our predictions. By contrast, the corresponding equation arising from symmetric (clamped) boundary conditions --- \eqn{f_m_clamped} --- predicts shifts that are $30 - 45$\% higher, and setbacks that are more than a factor of two larger. We conclude that \eqn{f_m_hybrid} and \eqn{b_0_kappa} can be used to reliably predict setbacks for classical guitars.

Finally, many luthiers effectively increase the value of $b$ (particularly for the wound bass strings) as the fret number increases to provide clearance for vibration amplitude at higher volume. In principle, this has an impact on the setbacks, but in practice these effects are not significantly large. Recall that the primary contribution to the saddle setback is the bending stiffness ($\Delta S \approx B_0\, X_0$), and does not depend on the height of the string above (e.g.) the twelfth fret. The nut setback depends primarily on the height of the string above the first fret, where the relief is minimal. (It's possible that the minor reliefs that we measured for the Japanese guitar discussed previously could account for the minor differences between our predictions and measurements.) Of course, it is possible to use both \sct{model} and \app{rms} --- along with a map of the effective value of $b$ for each string above each fret --- but it's unlikely that this will result in major changes to the values of either $\Delta S$ or $\Delta N$. 
