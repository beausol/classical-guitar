%%%%%%%%%%%%%%%%%%%%%%%%%%%%%%%%%%%%%%%%%%%%%%%%%%%%%%%%%%%%%%%%%%%%%%%%%%%%%%
%
% Section file included in main project file using \input{}
%
% Assumes that LaTeX2e macros and packages defined in cg_comp.sty are
%   available
%
%%%%%%%%%%%%%%%%%%%%%%%%%%%%%%%%%%%%%%%%%%%%%%%%%%%%%%%%%%%%%%%%%%%%%%%%%%%%%%

 \section{Experimental Estimate of the String Constant\label{sct:exp}}

 \begin{equation}
 \begin{split}
L_{n \ge 1}(y) &= \sqrt{\left(X_n + \Delta S\right)^2 + (b + c)^2 + y^2} \\
&\approx X_n + \Delta S + \frac{(b + c)^2 + y^2}{2\, X_n}\, .
 \end{split}
 \end{equation}

 \begin{equation}
 \begin{split}
L^\prime_{n \ge 1}(y) &= \sqrt{\left(X_0 - X_n + \Delta N\right)^2 + b^2 + y^2} \\
&\approx X_0 - X_n + \Delta N + \frac{b^2 + y^2}{2 \left(X_0 - X_n\right)}\, .
 \end{split}
 \end{equation}

 \begin{equation}
 \begin{split}
\lambda_n(y) &\approx \frac{1}{2\, X_0} \left[ \frac{(b + c)^2 + y^2}{X_n} + \frac{b^2 + y^2}{X_0 - X_n} - \frac{c^2}{X_0} \right] \\
&= \lambda_n(0) + \Delta \lambda_n(y) \, ,
 \end{split}
 \end{equation}
where $\lambda_n(0)$ is given by \eqn{lambda_n_approx}, and
 \begin{equation}
\Delta \lambda_n(y) \equiv \frac{1}{2 \left(\gamma_n - 1\right)}\, \left(\frac{\gamma_n\, y}{X_0}\right)^2\, .
 \end{equation}

Following the same approach we used to derive \eqn{quad_shift}, we can derive the change in the total shift due to both resonant length and linear mass density for a transverse displacement $y$. To second order in $y$, we find that
 \begin{equation}
\Delta \nu_n(y) \approx \frac{600}{\ln(2)}\, \frac{3 - 2 \gamma_n}{2 \left(\gamma_n - 1\right)}\, \left(\frac{\gamma_n\, y}{X_0}\right)^2\, .
 \end{equation}
We have plotted this expression for the first 12 frets and $y = 5$~mm in \fig{quad_shift_factory}. This shift is quite small compared to the experimental errors we'll obtain in the shifts due to tension, and we ignore it in what follows.

 \begin{figure}
  \centering
  \includegraphics[width=5.0in]{figures/quad_shift_factory}
  \caption{\label{fig:quad_shift_factory} Total frequency shift (in cents) due to resonant length and linear mass density for a transverse displacement of $y = 5$~mm. This shift is identical for each string, and should be smaller than the experimental errors we'll accumulate using our transverse displacement approach.}
 \end{figure}


% \begin{figure}
%  \centering
%  \begin{subfigure}[b]{0.8\textwidth}
%   \centering
%   \includegraphics[width=5.0in]{figures/norm_error_uncompensated}
%   \caption{Frequency shift for an uncompensated guitar}
%   \label{fig:norm_error_uncompensated}
%  \end{subfigure}
%  \par\vspace{0.25in}
%  \begin{subfigure}[b]{0.8\textwidth}
%   \centering
%   \includegraphics[width=5.0in]{figures/norm_error_factory}
%   \caption{Frequency shift for a factory guitar}
%   \label{fig:norm_error_factory}
%  \end{subfigure}
%  \caption{\label{fig:norm_error} Frequency shift (in cents) due to the fretted length $L_n$ for an uncompensated (a) and factory (b) Alhambra 8P guitar, for both zero and nonzero lateral displacement $y$.}
% \end{figure}
%
 \begin{figure}
  \centering
  \begin{subfigure}[b]{0.8\textwidth}
   \centering
   \includegraphics[width=5.0in]{figures/shift_data}
   \caption{Experimental data}
   \label{fig:shift_data}
  \end{subfigure}
  \par\vspace{0.25in}
  \begin{subfigure}[b]{0.8\textwidth}
   \centering
   \includegraphics[width=5.0in]{figures/delta_lambda}
   \caption{Calculated change in total string length $\mathcal{L}$}
   \label{fig:delta_l}
  \end{subfigure}
  \caption{\label{fig:exp_data} Frequency shift (in cents) (a) and change in total string length $\mathcal{L}$ (b) due to lateral displacement $y$.}
 \end{figure}
