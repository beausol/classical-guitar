%%%%%%%%%%%%%%%%%%%%%%%%%%%%%%%%%%%%%%%%%%%%%%%%%%%%%%%%%%%%%%%%%%%%%%%%%%%%%%
%
% Section file included in main project file using \input{}
%
% Assumes that LaTeX2e macros and packages defined in cg_comp.sty are
%   available
%
%%%%%%%%%%%%%%%%%%%%%%%%%%%%%%%%%%%%%%%%%%%%%%%%%%%%%%%%%%%%%%%%%%%%%%%%%%%%%%

 \section{Introduction and Background\label{sct:intro}}
Discuss initial~\cite{ref:byersweb} and ongoing work by G.\ Byers~\cite{ref:byers1996cgi}, and recent studies of steel-string guitars~\cite{ref:varieschi2010icf}. General reference on the physics of music and sound is \cite{ref:morse1981vas}.

Fundamental frequency of a string~\cite{ref:morse1981vsa}:
 \begin{equation} \label{eqn:f_0_def}
f_0 = \frac{1}{2\, L_0}\, \sqrt{\frac{T_0}{\mu_0}}\, ,
 \end{equation}
where $L_0$ is the length of the free (unfretted) string from the saddle to the nut, $T_0$ is the tension in the free string, and $\mu_0 \equiv M / L_0$ is the linear mass density of a free string of mass $M$.

