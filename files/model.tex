%%%%%%%%%%%%%%%%%%%%%%%%%%%%%%%%%%%%%%%%%%%%%%%%%%%%%%%%%%%%%%%%%%%%%%%%%%%%%%
%
% Section file included in main project file using \input{}
%
% Assumes that LaTeX2e macros and packages defined in cg_comp.sty are
%   available
%
%%%%%%%%%%%%%%%%%%%%%%%%%%%%%%%%%%%%%%%%%%%%%%%%%%%%%%%%%%%%%%%%%%%%%%%%%%%%%%

 \section{Simple Model of Guitar Intonation\label{sct:model}}

Fundamental frequency of a string~\cite{ref:morse1981vas,ref:morse1981vsa}:
 \begin{equation} \label{eqn:f_0_def}
f_0 = \frac{1}{2\, L_0}\, \sqrt{\frac{T_0}{\mu_0}}\, ,
 \end{equation}
where $L_0$ is the length of the free (unfretted) string from the saddle to the nut, $T_0$ is the tension in the free string, and $\mu_0 \equiv M / L_0$ is the linear mass density of a free string of mass $M$.

 \begin{equation} \label{eqn:f_0_stiff}
f_0 = \frac{1}{2\, L_0}\, \sqrt{\frac{T_0}{\mu_0}} \left[ 1 + B_0 + \left(1 + \frac{\pi^2}{8}\right) B_0^2 \right]\, ,
 \end{equation}
where $B_0$ is a ``string stiffness parameter.'' For a uniform string with a cylindrical cross section, $B_0$ given by~\cite{ref:morse1981vsb}
 \begin{equation} \label{eqn:b_def}
B_0 \equiv \sqrt{\frac{\pi\, R^4 E}{T_0\, L_0^2}}\, ,
 \end{equation}
where $R$ is the radius of the string and $E$ is Young's modulus (or the modulus of elasticity). For a typical nylon guitar string with $E \approx 2 - 4$~GPa, $T_0 \approx 50 - 70$~N, $R \approx 0.35 - 0.51$~mm, and $L_0 \approx 650$~mm, we have $B_0 \approx 0.007 - 0.026$.

Throughout this work, we will use \emph{cents} to describe small differences in pitch~\cite{ref:durfee2015pms}. One cent is one one-hundredth of a 12-TET half step, so that there are 1200~cents per octave. The difference in pitch between frequencies $f_1$ and $f_2$ is therefore defined as
 \begin{equation} \label{eqn:cents_def}
\Delta \nu \equiv 1200\, \log_2\left(\frac{f_2}{f_1}\right)\, .
 \end{equation}
We define $f \equiv (f_1 + f_2) / 2$ and $\Delta f \equiv f_2 - f_1$. Then
 \begin{equation} \label{eqn:cents_approx}
\Delta \nu = 1200\, \log_2\left(\frac{f + \Delta f / 2}{f - \Delta f /2}\right) \approx \frac{1200}{\ln 2}\, \frac{\Delta f}{f}\, ,
 \end{equation}
where the last approximation applies when $\Delta f \ll f$. An experienced guitar player can distinguish beat notes with a difference frequency of $\Delta f \approx 1$~Hz, which corresponds to 8~cents at $A_2$ ($f = 220$~Hz) or 5~cents at $E_4$ ($f = 329.63$~Hz).

Our model begins with the simple form of the fundamental frequency of a string given by \eqn{f_0_def}, and applies it to the frequency of a string pressed just behind the $n^\mathrm{th}$ fret:
 \begin{equation} \label{eqn:f_n_def}
f_n = \frac{1}{2\, L_n}\, \sqrt{\frac{T_n}{\mu_n}}\, ,
 \end{equation}
where (as shown in \red{Fig.~TBD}) $L_n$ is the \emph{resonant length} of the string from the saddle to fret $n$, and $T_n$ and $\mu_n$ are respectively the corresponding tension and linear mass density of the fretted string. We note that $T_n$ and $\mu_n$ depend on $\mathcal{L}_n$, the \emph{total} length of the fretted string from the saddle to the nut. Ideally, in the 12-TET equal-temperament system~\cite{ref:durfee2015pms},
 \begin{equation} \label{eqn:f_n_tet}
f_n = \gamma_n\, f_0\, , \qquad \textrm{(12-TET~ideal)}
 \end{equation}
where
 \begin{equation} \label{eqn:gamme_n_def}
\gamma_n \equiv 2^{n / 12}\, .
 \end{equation}
Therefore, the error interval expressed in cents is given by
 \begin{equation}\label{eqn:error_def}
 \begin{split}
\Delta \nu_n &= 1200\, \log_2\left( \frac{f_n}{\gamma_n\, f_0} \right) \\
&= 1200\, \log_2 \left( \frac{L_0}{\gamma_n\, L_n}\, \sqrt{\frac{\mu_0}{\mu_n}\, \frac{T_n}{T_0}}\, \right)\, .
 \end{split}
 \end{equation} 