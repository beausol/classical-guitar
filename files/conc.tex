%%%%%%%%%%%%%%%%%%%%%%%%%%%%%%%%%%%%%%%%%%%%%%%%%%%%%%%%%%%%%%%%%%%%%%%%%%%%%%
%
% Section file included in main project file using \input{}
%
% Assumes that LaTeX2e macros and packages defined in cg_comp.sty are
%   available
%
%%%%%%%%%%%%%%%%%%%%%%%%%%%%%%%%%%%%%%%%%%%%%%%%%%%%%%%%%%%%%%%%%%%%%%%%%%%%%%

 \section{Conclusion: The Recipeslabel{sct:conc}}

 In this work, we have constructed a model of classical guitar intonation that includes the effects of the resonant length of the fretted string, linear mass density, tension, and bending stiffness. We have described a simple experimental approach to estimating the increase in string tension arising from an increase in its length, and then the corresponding mechanical stiffness. This allows us to determine the saddle and nut positions needed to compensate the guitar for a particular string, and we propose a simple approach to find averages of these positions to accommodate a variety of strings. This ``mean'' method benefits further from temperament techniques --- such as harmonic tuning --- that can enhance the intonation of the classical guitar for particular musical pieces.

Our calculations have relied on \eqn{f_m_stiff}, which was derived by noting that the string was pinned to the saddle rather than clamped. We then separated the contributions to the frequency deviations from ideal values caused by fretting by measuring these differences using the definition of logarithmic ``cents'' given by \eqn{cents_def}. This approach has allowed us to determine Taylor series approximations to each of these contributions that are valid when the height of a string above each fret is small compared to the scale length. From these approximations, we have been able to create two arithmetic ``recipes'' --- the RMS method and a quadratic (in $B_0$) approximation --- that predict saddle and nut setbacks that enable the guitar to compensate for the frequency effects of fretting. Applying any one of these algorithms to a particular guitar design always begins with the same five steps:
\begin{enumerate}
    \item Determine the scale length of the guitar by doubling the distance between the inside edge of the nut and the center of the $12^\textrm{th}$ fret. 
    \item Using \fig{guitar_schematic} as a guide, carefully measure the values of $b$ and $c$. It is possible that the luthier has selected a saddle with vertical curvature, resulting in different values of $c$ for each string.
    \item Estimate the relief $\Delta y_{12}$ at the $12^\textrm{th}$ fret for each string. Measure the action (height) $y_{12}$ of the string above fret 12; then $\Delta y_{12} = y_{12} - b - c/2$ and we rescale the height of the saddle above the nut to $\tilde{c} = c + 2\, \Delta y_{12}$.
    \item Select a string set with values of $\kappa$ and $B_0$ listed in one of the derived physical properties tables in this paper, or follow the procedure developed in \sct{exp} to determine these quantities for a different string set.
    \item Referring to \fig{guitar_schematic} as a guide, choose a preferred value of the fretting distance $d$ to account for the size of the finger.
\end{enumerate}
The most accurate algorithm then adds one more step:
\begin{enumerate}
    \setcounter{enumi}{5}
    \item Using \eqn{rms_sol_quad} and the first line of \eqn{q_n_approx} --- with $L_0$, $L_n$, and $L_n^\prime$ computed using the exact forms of \eqn{l_0_def}, \eqn{l_n_def}, and \eqn{l_p_def}, respectively --- determine the saddle and nut setbacks for the selected string set.
\end{enumerate}

The quadratic approximation is based on \eqn{rms_sol_comp} (which chooses a 12-fret RMS average) and  is accessible using a hand calculator. In this case, Step 6 is updated and a Step 7 is added to adjust the setbacks for a nonzero value of $d$:
\begin{enumerate}
    \setcounter{enumi}{5}
    \item Compute the setbacks for $d = 0$~mm using \eqn{rms_sol_comp}:
    \begin{align*}
        \Delta S &\approx \left[ B_0 + \frac{3}{2} \left(1 + \pi^2\right) B_0^2 \right] X_0 + \frac{\kappa}{4\, X_0} \left( -9.848\, b^2 + 2\, b\, \tilde{c} + \tilde{c}^2 \right)\, , \nd \\
        \Delta N &\approx B_0^2\, X_0 - \frac{\kappa}{2\, X_0} \left( 5.633\, b^2 + b\, \tilde{c} \right)\, .
    \end{align*}
    % \item Compute the mean value of $Q$ for $d = 0$~mm using \eqn{qbar_approx}:
    % \begin{equation*}
    %     \overline{Q} \approx \frac{77.9\, b^2 + 35.6\, b\, \tilde{c} + 5.82\, \tilde{c}^{2}}{24\, X_0^2}\, .
    % \end{equation*}
    % Then compute the setbacks for $d = 0$~mm using \eqn{comp_approx}:
    % \begin{align*}
    %     \Delta S &\approx B_0\, X_0\, , \nd \\
    %     \Delta N &\approx - \kappa\, X_0\, \overline{Q} / 2\, ,
    % \end{align*}
    \item Choose a fretting distance $d$. Then adjust the setbacks computed in the previous step using
    \begin{align*}
        \Delta N^\prime &\approx \Delta N \left[ 1 + 13\, (d/X_0) + 280\, (d/X_0)^2 \right]\, , \nd \\
        \Delta S^\prime &\approx \Delta S + \Delta N \left[13\, (d/X_0) + 280\, (d/X_0)^2 \right]\, .
    \end{align*}
    As $d$ increases, the magnitude of the negative nut setback increases, and the saddle setback decreases because $\Delta N < 0$. 
\end{enumerate}
Of course, this quadratic approximation can be simplified further. For example, the terms proportional to $B_0^2$ can be neglected without much loss of accuracy (but without much gain in simplicity), yielding a corresponding linear approximation. Furthermore, the term in $\Delta S$ that is proportional to $\kappa$ can be neglected, but we recall from \sct{comp} that the resulting value of $\Delta S$ will be about  $15 - 20$\% too small.

These setback estimates can be averaged across the string set to design compensated nuts and saddles that should be relatively easy to fabricate. Nevertheless, we understand that high-end (concert) guitars that are likely to rely on one or two string sets (and the appropriate value of $d$ for one guitar player) will benefit from the full, more accurate treatment of individual string setbacks.

Throughout this work, it has become clear that the frequency errors at the first fret are difficult to eliminate. Increasing the magnitude of the nut setback will also increase the frequency errors at all other frets. The only obvious solution to this problem is also sacrilegious (and may result in string ``buzzing''): raise the height of the first fret to meet the string. Perhaps a clever luthier will discover a more acceptable method for accomplishing this objective.

We have placed the text of this manuscript (as well as the computational tools needed to reproduce our numerical results and all the graphs presented here) online~\cite{ref:github2021rgb} to invite comment and contributions from and collaboration with interested luthiers and musicians.