%%%%%%%%%%%%%%%%%%%%%%%%%%%%%%%%%%%%%%%%%%%%%%%%%%%%%%%%%%%%%%%%%%%%%%%%%%%%%%
%
% Section file included in main project file using \input{}
%
% Assumes that LaTeX2e macros and packages defined in cg_comp.sty are
%   available
%
%%%%%%%%%%%%%%%%%%%%%%%%%%%%%%%%%%%%%%%%%%%%%%%%%%%%%%%%%%%%%%%%%%%%%%%%%%%%%%

 \section{Conclusion\label{sct:conc}}

In this work, we have constructed a simple model of classical guitar intonation that includes the effects of the resonant length of the fretted string, linear mass density, tension, and bending stiffness. We have described a simple experimental approach to estimating the increase in string tension arising from an increase in its length, and then the corresponding mechanical stiffness. This allows us to determine the saddle and nut positions needed to compensate the guitar for a particular string, and we propose a simple approach to find averages of these positions to accommodate a variety of strings. This ``mean'' method benefits further from temperament techniques --- such as harmonic tuning --- that can enhance the intonation of the classical guitar for particular musical pieces.

In the future, we intend to conduct more precise experiments to measure $R$ for a reasonable set of guitar strings, and then to compare our theoretical predictions of compensated guitar/string frequency deviations with experimental results. We have placed the text of this manuscript (as well as the computational tools needed to reproduce our numerical results) online~\cite{ref:github2021rgb} so as to invite comment from and collaboration with interested musicians. 